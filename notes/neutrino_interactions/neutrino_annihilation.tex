
%\documentclass[12pt]{amsart}
%\usepackage{geometry} % see geometry.pdf on how to lay out the page. There's lots.
%\geometry{a4paper} % or letter or a5paper or ... etc
% \geometry{landscape} % rotated page geometry

\documentclass[12pt,letter]{article}

%\usepackage[latin1]{inputenc}
\usepackage{epsfig}
\usepackage{amsmath}
\usepackage{mdframed}
%\usepackage{amsfonts}
\usepackage{amssymb}
\usepackage{listings}
\usepackage{booktabs}
\usepackage{setspace}
\usepackage[colorlinks,citecolor=blue]{hyperref}
\usepackage{xcolor}
\usepackage[sort&compress,numbers]{natbib}
\usepackage{mathtools}
%\usepackage{natbibspacing}

\setlength{\oddsidemargin}{.46cm}
\setlength{\evensidemargin}{-0.46cm}
\setlength{\evensidemargin}{0.0cm}
\setlength{\textwidth}{15.5cm}
\setlength{\parskip}{1.0ex plus 0.6ex minus 0.6ex}
\setlength{\parindent}{1.0em} 
\setlength{\textheight}{22.5cm}
\setlength{\topmargin}{-2cm}
\setlength{\footskip}{2cm}
\setlength{\parindent}{0.0cm}

\def\aj{Astron. J.}
\def\apj{Astrophys. J.}
\def\apjl{Astrophys. J. Lett.}
\def\apjs{Astrophys. J. Supp. Ser. }
\def\aa{Astron. Astrophys. }
\def\aap{Astron. Astrophys. }
\def\araa{Ann.\ Rev. Astron. Astroph. }
\def\physrep{Phys. Rep. }
\def\mnras{Mon. Not. Roy. Astron. Soc. }
\def\mmsun{M_\odot}
\def\prl{Phys. Rev. Lett.}
\def\prd{Phys. Rev. D.}
\def\azh{Soviet Astron.}
\def\apss{Astrophys. Space Sci.}
\def\cqg{Class. Quantum Grav.}

\newcommand{\diver}[1]{\nabla \cdot \begin{bf}#1\end{bf}}
\newcommand{\pd}[2]{\frac{\partial #1}{\partial #2}}
\newcommand{\td}[2]{\frac{d #1}{d #2}}
\newcommand{\ld}[1]{\pd{#1}{t} + \begin{bf}v\end{bf}\cdot \nabla #1}
\newcommand{\thd}[3]{\biggl(\pd{#1}{#2}\biggr)_{#3}}
\newcommand{\avg}[1]{\langle #1 \rangle}
\newcommand{\sci}[2]{#1  \times 10^{#2}}
\newcommand{\script}[1]{\begin{cal} #1 \end{cal} }
\newcommand{\vect}[1]{\begin{bf} #1 \end{bf}}
\newcommand{\comm}[2]{\left[ #1 , #2 \right] }
\newcommand{\bra}[1]{\left \langle #1 \right |}
\newcommand{\ket}[1]{\left | #1 \right \rangle}
\newcommand{\braket}[2]{\langle #1 | #2 \rangle}
\newcommand{\code}[1]{\verb~#1~}
\newcommand{\vb}[1]{\begin{verbatim}#1\end{verbatim}}
\newcommand{\fsl}[1]{\ensuremath{\mathrlap{\!\not{\phantom{#1}}}#1}}

\usepackage{mathpazo}
\DeclareMathOperator{\diff}{d\!}
\newcommand{\modot}{M$_\odot$\hspace*{0.05cm}}
\def\apj{Astrophys. Journal}
\def\apjl{Astrophys. Journal Letters}
\def\aap{Astron. Astrophys.}

\setlength{\parskip}{.2cm}

\newcommand{\pdtwo}[3]{\frac{\partial #1}{\partial #2 \partial #3}} 
\newcommand{\pdtri}[4]{\frac{\partial #1}{\partial #2 \partial #3 \partial #4}} 
\newcommand{\grad}{\vec{\nabla}}
\newcommand{\ddiv}{\vec{\nabla}\cdot}
\newcommand{\intl}{\int\limits}
\newcommand{\msun}{M$_\odot$\ }
\newcommand{\mo}{\mathrm{M}_\odot }
\newcommand{\todo}[1]{{$\blacksquare$~\textbf{\color{blue}[TODO: #1]}}~$\blacksquare$}

% See the ``Article customise'' template for come common customisations

\title{A Note on Neutrino Interaction Rates}
\author{Luke Roberts}
\date{} % delete this line to display the current date

%%% BEGIN DOCUMENT
\begin{document}

\maketitle

Here I describe how to calculate the kernels for neutrino annihilation rates.  I use a metric with signature $\eta = (+,-,-,-)$.  In general, the spin summed and averaged squared matrix element is a Lorentz invariant that depends on contractions of the momenta of various particles.  For example, the weak matrix element is given by  
\begin{equation*}
\left\langle \left| M \right|^2 \right \rangle = 16 G_F^2 \left [ 
(C_V + C_A)^2 p^\mu_1 p_\mu^2 p^\nu_3 p_\nu^4
+(C_V - C_A)^2 p^\mu_2 p_\mu^3 p^\nu_1 p_\nu^4
-(C_V^2 - C_A^2) m_2 m_4 p^\nu_1 p_\nu^3
\right],
\end{equation*}
where the indices 1 and 3 represent the neutrino and antineutrino and 2 and 4 represent the electron and positron, respectively (see Bruenn, for instance).  This is also the matrix element for neutrino electron scattering, but with particles 1 and 2 is ingoing and 3 and 4 as outgoing particles.

We can also choose the standard decomposition of the matrix element in terms of the lepton tensor and something like the polarization function 
\begin{equation*}
\left\langle \left| M \right|^2 \right \rangle = \frac{G_F^2}{8}
L_{\mu \nu} \Lambda^{\mu \nu},
\end{equation*}
where 
\begin{eqnarray*}
L^{\mu \nu} &=& \textrm{Tr}\left[
(-\fsl{p}_1 + m_1) \gamma^\mu(1-\gamma^5) 
(-\fsl{p}_3 + m_3) \gamma^\nu(1-\gamma^5)  
\right] \\
&\approx& 8 \left[p_1^\mu p_3^\nu + p_1^\nu p_3^\mu 
- (p_1 \cdot p_3) \eta^{\mu \nu}   
+ i \epsilon^{\lambda \nu \kappa \mu} p_\lambda^1 p_\kappa^3
\right] \\
&=& 8 \left[2p_1^\mu p_1^\nu - p_1^\nu q^\mu - p_1^\mu q^\nu 
+ (p_1 \cdot q) \eta^{\mu \nu}   
- i \epsilon^{\lambda \nu \kappa \mu} p_\lambda^1 q_\kappa
\right].
\end{eqnarray*}
The baryon contribution to the matrix 
element is then given by 
\begin{eqnarray*}
\Lambda^{\mu \nu} &=& \textrm{Tr}\biggl[
(-\fsl{p}_2 + m_2) 
\left\{ C_V\gamma^\mu -C_A\gamma^\mu \gamma^5 
+ F_3\frac{i \sigma^{\mu \alpha}q_\alpha}{2 M_N} \right\}
\nonumber\\
&& \times \, (-\fsl{p}_4 + m_4)
\left \{ C_V\gamma^\nu -C_A\gamma^\nu \gamma^5 
- F_3\frac{i \sigma^{\nu \alpha}q_\alpha}{2 M_N} \right\}
\biggr] \\
&=& C_V^2 \Lambda^{\mu \nu}_V 
+ C_A^2 \Lambda^{\mu \nu}_A 
+ F_2^2 \Lambda^{\mu \nu}_F,
+ 2 C_V C_A \Lambda^{\mu \nu}_{VA} 
+ 2 F_2 C_V \Lambda^{\mu \nu}_{VT}, 
+ 2 F_2 C_A \Lambda^{\mu \nu}_{AT},
\end{eqnarray*}
where the $\Lambda$s are functions of $p_2^\mu$ and $q^\mu = p_4^\mu - p_2^\mu$.
Working through the spin sums in the neutral current case gives
\begin{eqnarray}
\Lambda^{\mu\nu}_V &=& 2 q_\alpha^2 \eta^{\mu \nu} 
+ 4(p_2^\nu q^\mu + p_2^\mu q^\nu) + 8 p_2^\mu p_2^\nu \\
\Lambda^{\mu\nu}_A &=& \Lambda^{\mu\nu}_V - 8 m_2^2 \eta^{\mu \nu} \\
\Lambda^{\mu\nu}_T &=& 2 q_\alpha^2 
\left(1 + \frac{q_\alpha^2}{4 m_2^2} \right)h^{\mu \nu}
-\frac{q_\alpha^2}{4 m_2^2} \Lambda^{\mu \nu}_V \\
\Lambda^{\mu\nu}_{VA} &=& 4 i \epsilon^{\mu \nu \alpha \beta}q_\alpha p_{2,\beta} 
\\
\Lambda^{\mu\nu}_{VT} &=& 2q_\alpha^2 h^{\mu \nu}
\\
\Lambda^{\mu\nu}_{AT} &=& 4 i \epsilon^{\mu \nu \alpha \beta}q_\alpha p_{2,\beta} , 
\end{eqnarray}
where $h^{\alpha \beta} = \eta^{\alpha \beta} - q^\alpha q^\beta/q_\mu^2$ is the
orthogonal projection tensor relative to $q^\alpha$.

\todo{Include charged current corrections to these results}

\section{Generic Decomposition of Tensors in Terms of $q^\mu$} 
Here, we partially follow Le Bellac on pages 113, 118, and 214. First, we choose to 
decompose the momentum transfer as $q_\mu = (q_0, 0, 0, q)$ and build a second 
vector orthogonal to $q_\mu$ in the time-z plane, $n_\mu = (q, 0, 0, q_0)$. We 
then define the transverse projector 
\begin{eqnarray}
P_{\mu \nu}^T &=& g_{\mu \nu} - \frac{q_\mu q_\nu}{q_\mu^2} - \frac{n_\mu n_\nu}{n^2} 
\nonumber\\
&=& g_{\mu \nu} + \frac{n_\mu n_\nu - q_\mu q_\nu}{q_\mu^2},
\end{eqnarray}
which picks out the portion of a vector orthogonal to $q_\mu$ and $n_\mu$. We 
can also define the projectors orthogonal to $q_\mu$ and $n_\mu$
\begin{eqnarray}
P^L_{\mu \nu} &=&  \frac{n_\mu n_\nu}{n_\mu^2} = -\frac{n_\mu n_\nu}{q_\mu^2}  \\
P^Q_{\mu \nu} &=&  \frac{q_\mu q_\nu}{q_\mu^2} \\
P^M_{\mu \nu} &=&  \frac{q_\mu n_\nu + q_\nu n_\mu}{q_\mu^2}
\end{eqnarray} 
The general decomposition of a vector in terms of these is given by 
\begin{equation}
f_\mu = f_q q_\mu + f_n n_\mu + \tilde f_\mu,
\end{equation}
where $\tilde f^\mu = P_T^{\mu \nu} f_\nu$. A symmetric tensor can be decomposed 
as 
\begin{equation}
h_{\mu\nu} = h_Q P^Q_{\mu \nu} + h_L P^L_{\mu \nu} 
+ h_M P^M_{\mu \nu} + \tilde h_{\mu \nu}.
\end{equation}
If there are no other preferred directions that enter into $h_{\mu \nu}$ (which is 
the case with the polarization tensors), we have $\tilde h_{\mu \nu} = h_T 
P^T_{\mu \nu}$, since $P^T$ is the only available transverse, symmetric tensor.
When current conservation holds for a polarization tensor, $q_\mu \Pi^{\mu \nu}
= 0$, we recover the standard decomposition of the vector part of the 
neutral current polarization tensor
\begin{equation}
\Pi_{\mu\nu}^{NC} = \Pi_L^{NC} P^L_{\mu \nu} + \Pi_T^{NC} P^T_{\mu \nu},
\end{equation}
which gives $\Pi^{NC}_L = -\frac{q_\mu^2}{q^2} \Pi^{NC}_{00}$ and $\Pi^{NC}_T =
-\Pi^{NC}_{22}$, similar to the relations used in previous works. In contrast, 
the charged current vector polarization tensor does not obey current conservation
and will have four separate pieces.  They can be recovered using 
\begin{eqnarray}
\Pi_Q &=& \frac{q^\mu q^\nu}{q_\mu^2} \Pi_{\mu \nu}\\
\Pi_L &=&-\frac{n^\mu n^\nu}{q_\mu^2} \Pi_{\mu \nu}\\ 
\Pi_M &=&-\frac{n^\mu q^\nu}{q_\mu^2} \Pi_{\mu \nu}\\
\Pi_T &=&-\Pi_{22}.
\end{eqnarray}
Because the polarization tensors can be expressed in terms of these tensors and 
scalar functions, we will need the contraction of these tensors with the lepton 
tensor to get the matrix element. Contracting with $q_\mu$ is straightforward, 
but contraction with $n_\mu$ is not obvious. First, we note that we can generically
write (assuming particle 1 is massless) 
\begin{eqnarray}
p_1^\mu = (p_1, -p_1 \sqrt{1-\mu_1^2}, 0, - p_1 \mu_1) \\
p_3^\mu = (p_1 - q_0, -p_1 \sqrt{1-\mu_1^2}, 0, - p_1 \mu_1 + q) \\
p_1 \cdot p_3 = -p_1 q_0  + q p_1 \mu_1 = -\frac{q_\alpha^2-m_3^2}{2},  
\end{eqnarray}
where the relation $p_1 \cdot p_3 = - (q_\alpha^2-m_3^2)/2$ is easily derived. 
Solving for $\mu_1$ and substituting into $p_1 \cdot n$ gives
\begin{eqnarray}
p_1 \cdot n &=& p_1 \left(q - \frac{q_0^2}{q} 
+ \frac{q_0 (q_\mu^2 - m_3^2)}{2 p_1 q} \right) \\
&=& - \frac{q_\alpha^2}{2 q}\left(E_1 + E_3 - \frac{q_0m_3^2}{q_\alpha^2} \right).
\end{eqnarray}  
With this result, one can show 
\begin{eqnarray}
L_{\mu \nu} q^\mu q^\nu &=& 0 \\
L_{\mu \nu} \eta^{\mu \nu} &=& 8 q_\alpha^2 \\
L_{\mu \nu} n^\mu n^\nu &=& 4 \left[ 4 (p_1 \cdot n)^2 - q_\alpha^2 \right]
\nonumber\\
&=& 8q_\alpha^2 \frac{4E_1E_3 + q_\alpha^2 - 2(E_1 + E_3) q_0 m_3^2/q_\alpha^2
+ q_0^2 m_3^4/q_\alpha^4}{2 q^2} = 8 q_\alpha^2 B, 
\end{eqnarray}
where the fraction in the last line reduces to the $B$ in the old paper when 
$m_3=0$. Then, we have
\begin{eqnarray}
L^{\mu \nu} P^T_{\mu \nu} &=& 8q_\alpha^2(B+1) \\  
L^{\mu \nu} P^L_{\mu \nu} &=&-8q_\alpha^2 B \\
L^{\mu \nu} P^Q_{\mu \nu} &=&0,
\end{eqnarray}
\todo{Still need to include relations for the mixed part}

\section{Scattering and Capture} 
The differential cross section for neutrino scattering or capture is 
\begin{eqnarray*}
\frac{d \sigma(E_1)}{V} &=& \frac{1}{2 E_1} d\textrm{LIPS}_3 \int d\textrm{LIPS}_2 
\int d\textrm{LIPS}_4 \\
&& \times \, (2 \pi)^4 \delta^{(4)}\left(p^\mu_1 - p^\mu_3 + p^\mu_2 - p^\mu_4 \right)
\left\langle \left| M \right|^2 \right \rangle f_2 (1-f_3) (1-f_4),
\end{eqnarray*}
where 
\begin{equation*}
d\textrm{LIPS}_i = \frac{d^3p_i}{(2\pi)^3 2 E_i} 
= \frac{p_i dE_i d \Omega_i}{(2\pi)^3 2} 
\end{equation*}
is the Lorentz invariant phase space element of species $i$.  Here, we define 
\begin{equation}
q^\mu = p_1^\mu - p_3^\mu=(q_0,0,0,-q),
\end{equation}
which is the energy momentum transfer from the neutrinos (or neutrino and lepton) to the other particle or particles involved in the interaction.  We can split this into neutrino dependent and neutrino independent pieces to get 
\begin{eqnarray*}
\frac{d \sigma(E_1)}{V} 
&=& \frac{G_F^2 \, d\textrm{LIPS}_3 (1-f_3)}
{16 E_1(1 - \exp((\mu_4 - \mu_2 - q_0)/T)} 
 L_{\mu \nu} \Pi^{\mu \nu}
\end{eqnarray*}
which defines the structure functions 
\begin{eqnarray*}
\Pi_{i}^{\mu \nu}(q^\alpha) &=& \int d\textrm{LIPS}_2 \int d\textrm{LIPS}_4
(2 \pi)^4 \delta^{(4)}\left(q^\alpha + p^\alpha_2 - p^\alpha_4 \right) 
(f_2-f_4) \Lambda_i^{\mu \nu}.
\end{eqnarray*} 
Using the momentum space delta function, we can integrate out the momenta of particle 4.  This gives
\begin{equation*}
p^\mu_4 = (E_4, \, 
-p_2 \cos \phi_2 \sqrt{1-\mu_2^2}, \,
-p_2 \sin \phi_2 \sqrt{1-\mu_2^2}, \,
-p_2 \mu_2 - q)
\end{equation*}
and 
\begin{equation*}
E_4^2 = E_2^2 - m_2^2 + m_4^2 + q^2 + 2 p_2 q \mu_2.  
\end{equation*}
Transforming the energy delta function to an angular delta function gives 
\begin{equation*}
\delta\left(q_0 + E_2 - \sqrt{E_2^2 - m_2^2 + m_4^2 + q^2 + 2 p_2 q \mu_2} \right) 
= \frac{E_4 \delta(\mu - \mu_0)}{p_2 q} 
\end{equation*} 
with $\mu_0 = \frac{q_\mu^2 + 2 E_2 q_0 + m_2^2 - m_4^2}{2 p_2 q}$.  Using this, we can write structure function as 
\begin{eqnarray*}
\Pi^{\mu \nu}_{i} &=& \frac{1}{(4 \pi)^2 q} \int_{0}^\infty dE_2 
\int d \Omega_2 \delta(\mu - \mu_0) \theta(E_2-e_{m})
(f_2 - f_4) \Omega_i^{\mu \nu},
\end{eqnarray*} 
where $\beta_m = 1 + (m_2^2 - m_4^2)/q_\mu^2$ and
\begin{equation}
e_{m}=-\beta \frac{q_0}{2} + \frac{q}{2} \sqrt{\beta^2 - \frac{4 m_2^2}{q_\mu^2}}.
\end{equation}  
We have the kinematic relations
\begin{eqnarray*}
q = \sqrt{p_1^2 + p_3^2 - 2 p_1 p_3 \mu_{13}} \\
q_0 = p_1 - p_3 \\ 
\end{eqnarray*}
which imply that $q_\mu^2 < 0$. Clearly, all of the polarizations can be expressed
in terms of $q^\mu$ and the tensors
\begin{equation}
\Pi^{\{a_k\}} = \frac{1}{(4\pi)^2 q} \int_0^\infty dE_2 \int d\Omega_2 
\Theta(\mu-\mu_0) (f_2 - f_4) p_2^{a_1}...p_2^{a_k}.
\end{equation}
This is basically the same as the PSTF formalism of Thorne 1981 and we can relate
different polarization tensors via 
\begin{equation}
q_\mu \Pi^{\mu \{a_k\}} = - \frac{\beta q_\mu^2}{2}\Pi^{ \{a_k\}}.
\end{equation}
\todo{Have checked this in Mathematica for the tensors we care about, but need 
to clarify if this is in fact general}

This relation comes in handy when trying to simplify the different parts of the 
polarization; when $\Lambda^{\mu \nu}_i$ contains $(p_2 \cdot q)$, it can be
replaced with $-\beta q_\mu^2/2$.  We now consider the expansion of $\Pi^{\mu \nu}$ 
in terms of the projections introduced in the previous section. The result is 
\begin{eqnarray}
\Pi_Q &=& \frac{\beta^2 q_\mu^2}{32 \pi q} \int_{e_m}^\infty dE_2(f_2 - f_4) \\
\Pi_L &=&-\frac{q_\mu^2}{32 \pi q^3} \int_{e_m}^\infty dE_2 (f_2 - f_4) 
(2 E_2 + \beta q_0)^2 \\
\Pi_M &=&-\frac{\beta q_\mu^2}{32 \pi q^2} \int_{e_m}^\infty dE_2 (f_2 - f_4) 
(2 E_2 + \beta q_0) \\
\Pi_T &=& - \frac{1}{2}\Pi_L 
+ \left(\frac{2 m_2^2}{\beta^2 q_\alpha^2} - \frac{1}{2}\right)\Pi_Q.
\end{eqnarray}
We then find 
\begin{eqnarray}
\Pi^\nu &=& -\frac{2 q^\nu}{\beta q_\alpha^2} \Pi_Q 
- \frac{2 n^\nu}{\beta q_\alpha^2} \Pi_M \\
\Pi &=& \frac{4 }{\beta^2 q_\alpha^2} \Pi_Q 
\end{eqnarray}


\subsection{Pieces of the NC polarization}
The most complicated piece of the polarization is the vector piece, which can be 
expressed in terms of a Longitudinal piece and a Transverse piece 
\begin{eqnarray}
\Pi_L^{vv} &=& -\frac{n^\mu n^\nu}{q_\alpha^2} \Pi^{vv}_{\mu \nu} 
= 8 \Pi_L + 2 q_\alpha^2 \Pi = 8(\beta^{-2} \Pi_Q + \Pi_L) \nonumber \\
&=& -\frac{q_\mu^2}{4 \pi q^3} \int_{e_m}^\infty dE_2 (f_2 - f_4) 
\left[(2 E_2 + \beta q_0)^2 - q^2 \right] \\
\Pi_T^{vv} &=& 2 q_\alpha^2 \Pi + 8 \Pi_T = 8 (\beta^{-2} \Pi_Q + \Pi_T), 
\end{eqnarray}
the other projections are zero due to current conservation. 
\todo{Understand why the longitudinal expression differs from Sanjay's old work 
by a sign and a factor of 2} Therefore, the vector polarization is 
\begin{equation}
\Pi^{vv}_{\mu \nu} = \Pi^{vv}_L P^L_{\mu \nu} + \Pi^{vv}_T P^T_{\mu \nu}
\end{equation}
We can also see that 
\begin{equation}
\Pi^{aa}_{\mu \nu} = \Pi^{vv}_L P^L_{\mu \nu} + \Pi^{vv}_T P^T_{\mu \nu} 
+ \eta_{\mu \nu} \Pi_A
\end{equation}
with 
\begin{equation}
\Pi_A =-8 m_2^2 \Pi = -\frac{32 m_2^2}{\beta^2 q_\alpha^2} \Pi_Q
= \frac{m_2^2}{\pi q} \int_{e_m}^\infty dE_2 (f_2 - f_4). 
\end{equation}
\todo{Once again this differs by a sign and a factor of 2} 
The vector-axial tensor obviously does not depend on our symmetric projections.
We also note that the portion of $\Pi_\mu$ proportional to $q_\mu$ is killed off
by the completely anti-symmetric tensor. Therefore,
\begin{equation}
\Pi^{VA}_{\mu \nu} = \Pi^{AT}_{\mu \nu} 
=- 8 i \epsilon^{\mu \nu \alpha \beta} q_\alpha n_\beta \frac{1}{\beta q_\lambda^2}
\Pi_M.
\end{equation}
The other two terms are just proportional to $\Pi$.


\section{Annihilation}
The differential annihilation cross section for neutrino annihilation is 
\begin{eqnarray*}
\frac{d \sigma(E_1)}{V} &=& \frac{1}{2 E_1} d\textrm{LIPS}_3 \int d\textrm{LIPS}_2 
\int d\textrm{LIPS}_4 \\
&& \times \, (2 \pi)^4 \delta^{(4)}\left(p^\mu_1 + p^\mu_3 - p^\mu_2 - p^\mu_4 \right)
\left\langle \left| M \right|^2 \right \rangle f_3 (1-f_2)(1-f_4).
\end{eqnarray*}
If we define the energy momentum transfer from the neutrinos to the leptons as
\begin{equation*}
q^\mu = p_1^\mu + p_3^\mu = (q_0,0,0,-q)
\end{equation*}
and employ the matrix element given above, we can rewrite the differential cross section as 
\begin{eqnarray*}
\frac{d \sigma(E_1)}{V} &=& \frac{1}{2 E_1}  d\textrm{LIPS}_3 f_3 
p^1_\mu p^3_\nu \\
&&\times \, 16 G_F^2\left( 
 (C_V+C_A)^2 \Lambda_{(1)}^{\mu \nu}
+(C_V-C_A)^2 \Lambda_{(2)}^{\mu \nu}
-(C_V^2-C_A^2) \Lambda_{(3)}^{\mu \nu} \right)
\end{eqnarray*}
which defines 
\begin{eqnarray*}
\Lambda_{(1)}^{\mu \nu}(q^\alpha) = \int d\textrm{LIPS}_2 \int d\textrm{LIPS}_4
(2 \pi)^4 \delta^{(4)}\left(q^\alpha - p^\alpha_2 - p^\alpha_4 \right) 
(1-f_2)(1-f_4)p^\mu_2 p^\nu_4 \nonumber\\
\Lambda_{(2)}^{\mu \nu}(q^\alpha) = \int d\textrm{LIPS}_2 \int d\textrm{LIPS}_4
(2 \pi)^4 \delta^{(4)}\left(q^\alpha - p^\alpha_2 - p^\alpha_4 \right) 
(1-f_2)(1-f_4) p^\nu_2 p^\mu_4 \nonumber\\
\Lambda_{(3)}^{\mu \nu}(q^\alpha) = \int d\textrm{LIPS}_2 \int d\textrm{LIPS}_4
(2 \pi)^4 \delta^{(4)}\left(q^\alpha - p^\alpha_2 - p^\alpha_4 \right) 
(1-f_2)(1-f_4) \eta^{\mu \nu} m_2 m_4.
\end{eqnarray*}
We now consider $\Lambda_{(1)}$, everything else generalizes from this easily.  
First, we use the momentum space delta function to get rid of the integration over the momentum of particle 4.  This demands that  
\begin{equation*}
p_4^\mu = \left(E_4, p_3 \cos \phi_3 \sqrt{1-\mu_3^2}, 
p_3 \cos \phi_3 \sqrt{1-\mu_3^2}, p_3 \mu_3 - q\right),  
\end{equation*}
which gives 
\begin{equation*}
E_4^2 = E_3^2 + q^2 - 2 p_3 q \mu_3.
\end{equation*}
Then, we transform from a delta function in energy to a delta function in $\mu_3$ using the relation
\begin{equation*}
\delta(g(x)) = \frac{\delta(x-x_0)}{|g'(x_0)|}.
\end{equation*} 
This gives 
\begin{equation*}
\delta\left(q_0 - E_2 - \sqrt{E_2^2 + q^2 - 2 p_2 q \mu_2}\right) 
= \frac{2 E_4 \delta(\mu - \mu_0)}{p_2 q}    
\end{equation*}
with $\mu_0 = \frac{2 E_2 q_0 - q_\mu^2}{2 p_2 q}$.  Combining these results gives,
\begin{equation*}
\Lambda_{(1)}^{\mu \nu} = \frac{2}{(4 \pi)^2 q}\int dE_2 d\Omega_2 
\delta(\mu-\mu_0) (1-f_2)(1-f_4)p^\mu_2 p^\nu_4,
\end{equation*}
where the $E_4$ provided by the delta function is canceled by the one provided by the LIPS.  Because the integral over $\mu_2$ only runs from -1 to 1, the delta function puts limits on the the integration over energy.  Setting $\mu_2$ equal to 1 and -1 results and solving for $E_2$ results in the limits 
\begin{equation*}
E_{min/max} = \frac{q_0}{2} \pm \frac{q}{2} \sqrt{1 - \frac{4 m^2}{q_\mu^2}}.
\end{equation*} 

\subsection{No Final State Blocking}
When final state blocking is neglected (i.e. $f_2 = f_4 = 0$) these equations simplify significantly.  First, we note that $\Lambda_{(1)} = \Lambda_{(2)}$ and the $\Lambda$s become symmetric tensors.  We can also write 
\begin{equation*}
\Lambda_{(1)}^{\mu \nu} = 
h(q^\alpha q_\alpha) q^\mu q^\nu + g(q^\alpha q_\alpha) \eta^{\mu \nu},
\end{equation*}
since there is no preferred direction in the integration.  

First, we calculate 
\begin{equation*}
\Lambda_{(3)}^{\mu \nu} = \frac{m_1 m_2 \eta^{\mu \nu}}{2^3 \pi^2 q} 
\int_{E_{min}}^{E_{max}} dE_2  \int d\Omega_2 
\delta(\mu_2 - \mu_0) = \eta^{\mu \nu} \frac{m_1 m_2}{2^2 \pi} 
\sqrt{1 - \frac{4m^2}{q_\alpha^2}}.
\end{equation*}

Next we calculate $\Lambda_{(1)}^{11} = -g$.  
\begin{eqnarray*}
\Lambda_{(1)}^{11} &=& \frac{2}{(4 \pi)^2 q}\int_{E_{min}}^{E_{max}} dE_2 
d\Omega_2 \delta(\mu-\mu_0) p^2_2 
(-\cos \phi \sqrt{1-\mu_2^2})(\cos \phi \sqrt{1-\mu_2^2}) \\
&=& -\frac{2 \pi}{(4 \pi)^2 q}\int_{E_{min}}^{E_{max}} dE_2 
p^2_2(1-\mu_0^2) \\
&=& -\frac{2 \pi}{(4 \pi)^2 q}\int_{E_{min}}^{E_{max}} dE_2 
\left[p^2_2 - \frac{1}{q^2} 
\left(E_2^2 q_0^2 + q_\alpha^4/4 - E_2 q_0 q_\alpha^2\right) \right] \\
&=& -\frac{\pi q_\alpha^2}{3(4 \pi)^2}
\left(1 - \frac{4m^2}{q_\alpha^2}\right)^{3/2} \\
&=& -\frac{q_\alpha^2}{48 \pi}
\left(1 - \frac{4m^2}{q_\alpha^2}\right)^{3/2}
\end{eqnarray*}

Finally, we can calculate $\Lambda_{(1)}^{00} = h q_0^2 + g$ and get $h$.  
\begin{eqnarray*}
\Lambda_{(1)}^{00} &=& \frac{2}{(4 \pi)^2 q}\int_{E_{min}}^{E_{max}} dE_2 
d\Omega_2 \delta(\mu-\mu_0)  
E_2(q_0 - E_2) \\
&=& \frac{4 \pi}{(4 \pi)^2 q}\int_{E_{min}}^{E_{max}} dE_2 
E_2(q_0 - E_2) \\
&=& \frac{1}{48 \pi q_\alpha^2 }
\left(4 m^2 q^2 + q_\alpha^2(2q_0^2 + q_\alpha^2) \right)  
\sqrt{1-\frac{4m^2}{q_\alpha^2}}
\end{eqnarray*}  
which gives 
\begin{equation*}
h = \frac{q_\alpha^2 + 2 m^2}{24 \pi \, q_\alpha^2} \sqrt{1 - \frac{4m^2}{q_\alpha^2}}.
\end{equation*}
Therefore, we have 
\begin{eqnarray*}
\Lambda_{(1,2)}^{\mu\nu} &=&
\frac{q_\alpha^2 + 2 m^2}{24 \pi \, q_\alpha^2} \lambda
q^\mu q^\nu
+  \frac{q_\alpha^2}{48 \pi} 
\lambda^3 \eta^{\mu \nu}, \\
\Lambda_{(3)}^{\mu\nu} &=& \frac{m^2}{4 \pi} 
\lambda \eta^{\mu \nu},
\end{eqnarray*}
with $\lambda = \sqrt{1 - 4m^2/q_\alpha^2}$ and the kinematic constraints that $q_\alpha^2 > 4m^2$ and $q_0>q$ (Note: I have checked $\Lambda^{33}$ by hand and it agrees with this result, see the mathematica notebook).  Before considering the differential cross section, it is useful to look at some kinematic quantities in terms of the neutrino momenta 
\begin{eqnarray*}
q &=& \sqrt{p_1^2 + p_3^2 + 2 p_1 p_3 \mu_{13}} \\
q_0 &=& p_1 + p_3 \\
q_\mu^2 &=& q_0^2 - q^2 = 2 p_1 p_3(1-\mu_{13}) \\ 
p_1 \cdot p_3 &=& p_1 p_3 (1-\mu_{13}) = q_\mu^2/2 \\
p_1 \cdot q &=& p_1 \cdot p_3 = q_\mu^2/2
\end{eqnarray*}
The total cross section is now given by 
\begin{eqnarray*}
\frac{\sigma(E_1)}{V} &=& \frac{16 G_F^2}{4 (2 \pi)^3 E_1}  
\int \frac{d^3p_3}{E_3} f_3 \frac{q_\alpha^2 \lambda}{4 \pi} \\
&&\times \, \left[ 
 (C_V^2+C_A^2) \frac{q_\alpha^2 + 2 m^2}{3}
+(C_V^2+C_A^2) \frac{q_\alpha^2 - 4 m^2}{6} 
+(C_A^2-C_V^2) m^2 \right] \\
\frac{\sigma(E_1)}{V} &=& \frac{G_F^2}{(2 \pi)^4}  
\int d\Omega_3 \int_{\frac{2 m^2}{E_1 (1-\mu_3)}}^\infty dE_3 \frac{E_3}{E_1}
 q_\alpha^2 \lambda \, \left[(C_V^2+C_A^2) q_\alpha^2 
 - (C_V^2-C_A^2) 2 m^2 \right]f_3.
\end{eqnarray*}
The lower limit on the energy integral comes from demanding that $q_\mu^2>4m^2$.  Clearly, the differential cross section can never be negative if the kinematic constraint $q_\mu^2 > 4m^2$ is obeyed.    
\todo{Check that the factors in front of the differential cross section are correct.  Units work out Look at old notes for this.}
The total annihilation rate will be 
\begin{eqnarray*}
\dot N &=& \int \frac{d^3p_1}{(2\pi)^3} \frac{\sigma(E_1)}{V} \\
&=& \frac{G_F^2}{(2 \pi)^7}  
\int d\Omega_1 \int d\Omega_3 \int_0^\infty dE_1
\int_{\frac{2 m^2}{E_1 (1-\mu_3)}}^\infty dE_3 \, E_3 E_1
 q_\alpha^2 \lambda \, \left[(C_V^2+C_A^2) q_\alpha^2 
 - (C_V^2-C_A^2) 2 m^2 \right]f_3 \\
 &=& \frac{G_F^2}{(2 \pi)^7}  
\int d\Omega_1 \int d\Omega_3 \int_0^\infty dE_1
\int_{\frac{2 m^2}{E_1 (1-\mu_3)}}^\infty dE_3 \, E_1^2 E_3^2 (1-\mu_{13})
\lambda \, \\
&& \times \left[(C_V^2+C_A^2) E_1 E_3 (1-\mu_{13}) 
 - (C_V^2-C_A^2) 2 m^2 \right]f_3 
\end{eqnarray*}
Aside from the lower limit on the $E_3$ integral and the factor of $\lambda$, this has the same form as the Ruffert et al. result.  

\section{Spinor Technology and Matrix Elements in RMF Theory}
For Walecka type field theories, we generally have a Lagrangian of the the form 
\begin{eqnarray}
\script{L} &=& \sum_B \bar \Psi_B (-i \gamma^\mu \partial_\mu - M_B 
+ g_{\sigma B} \sigma - g_{\omega B} \gamma^\mu \omega_\mu)\Psi_B 
+ \frac{1}{2} m_\omega^2 \omega_\mu \omega^\mu 
- \frac{1}{4} W_{\mu \nu} W^{\mu \nu} \\
&& - \partial_\mu \sigma \partial^\mu \sigma 
- \frac{1}{2} m_\sigma^2 \sigma^2 - U(\sigma)
\end{eqnarray}
where we have subsumed all of the vector fields into one vector field 
$\omega_\mu$ and the mass term schematically accounts for all of these. We also
define $W_{\mu \nu} = \partial_\mu \omega_\nu - \partial_\nu \omega_\mu$.  We 
can then write down the field equations from 
\begin{eqnarray}
\partial_\mu \left(\pd{\script{L}}{(\partial_\mu \psi)}\right) =  
\pd{\script{L}}{\psi} 
\end{eqnarray}
We also assume that the fields are homogenous and stationary and also assume that the vector field has zero spatial components.  For the $\sigma$ field, we are left with
\begin{eqnarray}
\partial_\mu \partial^\mu \sigma - m_\sigma^2 \sigma = -\sum_B g_{\sigma B} \bar \Psi_B \Psi_B + \td{U(\sigma)}{\sigma} \\
m_\sigma^2 \sigma = -\td{U(\sigma)}{\sigma} + \frac{1}{V}\sum_B g_{\sigma B}\avg{ \bar \Psi_B \Psi_B}
\end{eqnarray}
In the second line spatial homogeneity has been assumed and the brackets denote the thermal average (including an integration over space.
For the $\omega$ field we are left with
\begin{eqnarray}
m_\omega^2\omega_0 = \frac{1}{V} \sum_B g_{\omega B} \avg{ \bar \Psi_B \gamma^0 \Psi_B} \\
m_\omega^2\omega_0 = \frac{1}{V} \sum_B g_{\omega B} \avg{ \Psi_B^\dagger \Psi_B}
\end{eqnarray}
Where the second line was obtained from the properties of the gamma matrices (i.e. $\gamma^0 \gamma^0 = 1$).  Clearly, the thermally averaged quantity is just the conserved charge portion of the baryon current.  
The field equation for the baryon field is given by
\begin{eqnarray}
(-i \gamma^\mu \partial_\mu - M_B + g_{\sigma B} \sigma - g_{\omega B} \gamma^\mu \omega_\mu)\Psi_B = 0 \\
\bar \Psi_B(i \gamma^\mu \ \overleftarrow{\partial}_\mu - M_B + g_{\sigma B} \sigma - g_{\omega B} \gamma^\mu \omega_\mu) = 0
\end{eqnarray}

It is now useful to expand the baryon fields in terms of creation and annihilation operators $a_s$ and $b_s$ and the standard four-component spinners $u_s$ and $v_s$ as 
\begin{eqnarray}
\Psi(x) = \sum_{s=+,-} \int \frac{d^3p}{(2 \pi)^3 E} \left(a_s(\vec{p}) u_s(\vec{p}) e^{ip\cdot x} + b^\dagger_s(\vec{p}) v_s(\vec{p}) e^{-ip\cdot x} \right ) \\ 
\bar \Psi(x) =  \sum_{s=+,-} \int \frac{d^3p}{(2 \pi)^3 E} \left(a_s^\dagger(\vec{p}) \bar u_s(\vec{p}) e^{-ip\cdot x} + b_s(\vec{p}) \bar v_s(\vec{p}) e^{ip\cdot x} \right )
\end{eqnarray}
Using the field equation for the baryon fields with the above expansions results in
\begin{eqnarray}
( \gamma^\mu p_\mu - M_*  - g_{\omega B} \gamma^\mu \omega_\mu)u_s(\vec{p}) = 0 \\
\bar u_s(\vec{p})  ( \gamma^\mu p_\mu - M_*  - g_{\omega B} \gamma^\mu \omega_\mu)= 0 \\
( -\gamma^\mu p_\mu - M_* - g_{\omega B} \gamma^\mu \omega_\mu)v_s(\vec{p}) = 0 \\
\bar v_s(\vec{p}) ( -\gamma^\mu p_\mu - M_* - g_{\omega B} \gamma^\mu \omega_\mu)= 0 
\end{eqnarray}
Defining $\tilde p_\mu = p_\mu \mp g_{\omega B} \omega_\mu$ makes these exactly the same as the dispersion relations for a free field theory, except with $\tilde p_\mu$ instead of $p_\mu$.  Determining the eigenvalues of this equation results in the dispersion relation
\begin{equation}
\script{E}^\pm = g_{\omega B} \omega_0 \pm \sqrt{p^2 + M^2_*} \equiv g_{\omega B} \omega_0 \pm E
\end{equation}
Additionally, this implies that the Gordon identities and other properties of the spinors are unaltered from the free field case aside from the aforementioned replacement.  Therefore, we have 
\begin{eqnarray}
\sum_{s=\pm} \bar u_s(\vec{p}) u_s(\vec{p}) &=& -\fsl{\tilde p} + M_* \\
\sum_{s=\pm} \bar v_s(\vec{p}) v_s(\vec{p}) &=& -\fsl{\tilde p} - M_*.
\end{eqnarray}
This in turn implies that the mean field matrix element can be found from the free matrix element by the replacement $p_\mu \rightarrow \tilde p_\mu$.  Additionally, I think the Lorentz invariant phase space factor should have $E$ in the denominator instead of $\script{E}$.
\end{document}







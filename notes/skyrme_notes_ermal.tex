\documentclass[preprint,prc,preprintnumbers,superscriptaddress,amsmath,amssymb,floatfix]{revtex4-1}
\usepackage{graphicx,graphics,setspace,epsfig,color}
\usepackage[letterpaper,dvips,width=7.5in,height=8.5in,includemp=false]{geometry}
\usepackage[vcentermath]{}
\usepackage{epsf}
\usepackage{url}
\usepackage{amscd}
\usepackage{amsmath}
\usepackage[cm]{fullpage}
\usepackage{amsmath}
\usepackage{amssymb}
\usepackage{mathtools}
\usepackage{mdframed}
\usepackage{textcomp}
\usepackage{subcaption}
\usepackage{slashed}
\usepackage{url}
\usepackage{hyperref}
\usepackage[section]{placeins}
\usepackage{float}
\usepackage[active]{srcltx}
\usepackage{color}
\usepackage{inputenc}
\usepackage[T4,T1]{fontenc}
\usepackage{mathtext}
\usepackage{comment}
\usepackage{stmaryrd}
\newcommand{\CZ}{{\cal Z}}
\newcommand{\CL}{{\cal L}}

\setlength{\textheight}{8.85in}
\setlength{\topmargin}{.01in}
% ----------------------------------------------------------------
\vfuzz2pt % Don't report over-full v-boxes if over-edge is small
\hfuzz2pt % Don't report over-full h-boxes if over-edge is small

\long\def\symbolfootnote[#1]#2{\begingroup%
\def\thefootnote{\fnsymbol{footnote}}\footnote[#1]{#2}\endgroup} 
\newcommand\qslash {\slashed{q}}
\newcommand\pslash {\slashed{p}}
\newcommand\dslash {\slashed{\partial}}
\newcommand\Dslash {\slashed{D}}
\newcommand{\half}{\frac{1}{2}}
\newcommand{\be}{\begin{equation}}
\newcommand{\ee}{\end{equation}}
\newcommand\beq{\begin{eqnarray}}
\newcommand\eeq{\end{eqnarray}} 
\newcommand\nnnlo {N$^3$LO}
\newcommand{\E}


\def\apjl{ApJL }
\def\aj{AJ }
\def\apj{ApJ }
\def\apjs{ApJ  Supplement}
\def\pasp{PASP }
\def\spie{SPIE }
\def\apjs{ApJS }
\def\araa{ARAA }
\def\aap{A\&A }
\def\nat{Nature }
\usepackage{hyperref}
 

\begin{document}

\title{EOS Constraints}

\author{Ermal Rrapaj}
\email{ermal@uw.edu}
\affiliation{Department of Physics, University of Washington, Seattle, WA}
\affiliation{Institute for Nuclear Theory, University of Washington, Seattle, WA}
\begin{abstract}

\end{abstract}
\section{Skyrme interaction - Finite Temperature}
\subsection{Potential Matrix Element}
Interaction Matrix:
\begin{equation}
 \begin{split}
  V_{ij}=&t_0(1+x_0 P_{\sigma})\delta(\mathbf{r}_{ij})+\frac{1}{2}t_1(1+x_1P_{\sigma}) \frac{1}{\hbar^2}[\overleftarrow{p^{\dagger}}^2_{ij}\delta(\mathbf{r}_{ij})+\delta(\mathbf{r}_{ij})\overrightarrow{p}^2_{ij}]\\
  &+t_2(1+x_2P_{\sigma})\frac{1}{\hbar^2}\overleftarrow{\mathbf{p}^{\dagger}}_{ij}\cdot \delta(\mathbf{r}_{ij})\overrightarrow{\mathbf{p}}_{ij}+\frac{1}{6}t_3(1+x_3P_{\sigma})\rho^{\alpha}(\mathbf{r})\delta(\mathbf{r}_{ij})\\
  &+\frac{i}{\hbar^2}W_0(\mathbf{\sigma}_i+\mathbf{\sigma}_j)\cdot \overleftarrow{\mathbf{p}^{\dagger}}_{ij} \times \delta(\mathbf{r}_{ij})\overrightarrow{\mathbf{p}}_{ij}\\
  &+\frac{1}{4}t_4(1+x_4 P_{\sigma})\frac{1}{\hbar^2}[ \overleftarrow{p^{\dagger}}^2_{ij}\rho^{\beta}(\mathbf{r})\delta(\mathbf{r}_{ij})+\delta(\mathbf{r}_{ij})\rho^{\beta}(\mathbf{r})\overrightarrow{p}^2_{ij}]\\
  &+t_5(1+x_5P_{\sigma})\frac{1}{\hbar^2}\overleftarrow{\mathbf{p}}_{ij}\cdot \rho^{\gamma}(\mathbf{r})\delta(\mathbf{r}_{ij})\overrightarrow{\mathbf{p}}_{ij}
 \end{split}
\end{equation}
where, $\mathbf{r}_{ij}=\frac{\mathbf{r}_i-\mathbf{r}_j}{2},\ \mathbf{r}=\frac{\mathbf{r}_i+\mathbf{r}_j}{2},\ P_{\sigma}= \frac{1}{2}(1+\mathbf{\sigma}_1\cdot \mathbf{\sigma}_2),\ \mathbf{p}_{ij}=-i\hbar \frac{\boldmath{\bigtriangledown}_i-\mathbf{\bigtriangledown}_j}{2},\ \rho(\mathbf{r})=\rho_p(\mathbf{r})+\rho_n(\mathbf{r})$
 and the arrows show the direction on which the momentum operators act.
 
 
\subsection{Energy Density}
Assume time reversal invariance. The Energy of the ground state can be written as integration over an energy density functional, $E_{HFB} = \int d^r  \varepsilon_{HFB} $
\begin{equation}
 \begin{split}
  \varepsilon_{HFB} &= \sum_{t=n,p}\frac{\hbar^2}{2M_q}\tau_q+\frac{1}{2}t_0\Big[(1+\frac{1}{2}x_0)\rho^2-(\frac{1}{2}+x_0)\sum_{q=n,p}\rho^2_q\Big]\\
  &+\frac{1}{4}t_1\Big[ (1+\frac{1}{2}x_1)(\rho \tau+\frac{3}{4}(\triangledown \rho)^2)-(\frac{1}{2}+x_1)\sum_{q=n,p}(\rho_q \tau_q+\frac{3}{4}(\triangledown \rho_q)^2)\Big]\\
  &+\frac{1}{4}t_2\Big[ (1+\frac{1}{2}x_2)(\rho \tau+\frac{3}{4}(\triangledown \rho)^2)+(\frac{1}{2}+x_2)\sum_{q=n,p}(\rho_q \tau_q-\frac{1}{4}(\triangledown \rho_q)^2)\Big]\\
  &+\frac{1}{12}t_3\rho^{\alpha}\Big[(1+\frac{1}{2}x_3)\rho^2-(\frac{1}{2}+x_3)\sum_{q=n,p}\rho_q^2\Big]\\
  &+\frac{1}{4}t_4\Big[ (1+\frac{1}{2}x_4)(\rho \tau+\frac{3}{4}(\triangledown \rho)^2)-(\frac{1}{2}+x_4)\sum_{q=n,p}(\rho_q \tau_q+\frac{3}{4}(\triangledown \rho_q)^2)\Big]\rho^{\beta}\\
 &+\frac{\beta}{8}t_4\Big[(1+\frac{1}{2}x_4)\rho(\triangledown \rho)^2-(\frac{1}{2}+x_4)\mathbf{\triangledown}\rho \cdot \sum_{q=n,p}\rho_q\mathbf{\triangledown}\rho_q\Big]\rho^{\beta-1}\\
 &+\frac{1}{4}t_5\Big[ (1+\frac{1}{2}x_5)(\rho \tau+\frac{3}{4}(\triangledown \rho)^2)+(\frac{1}{2}+x_5)\sum_{q=n,p}(\rho_q \tau_q-\frac{1}{4}(\triangledown \rho_q)^2)\Big]\rho^{\gamma}\\
 &-\frac{1}{16}(t_1x_1+t_2x_2)J^2+\frac{1}{16}(t_1-t_2)\sum_{q=n,p}J^2_q\\
 &-\frac{1}{16}(t_4 x_4\rho^{\beta}+t_5x_5\rho^{\gamma})J^2+\frac{1}{16}(t_4\rho^{\beta}-t_5\rho^{\beta})\sum_{q=n,p}J^2_q\\
 &+\frac{1}{2} W_0 (\mathbf{J}\cdot \mathbf{\triangledown}\rho+\sum_{q=n,p}\mathbf{J}_q\cdot \mathbf{\triangledown}\rho_q)
 \end{split}
\end{equation}
where,
\begin{equation}
 \begin{split}
  \rho &= 2\int \frac{d^3 k}{(2 \pi \hbar)^3} n(k)\\
  \tau &= 2\int \frac{d^3 k}{(2 \pi \hbar)^3} k^2 n(k)\\
  \mathbf{J}&= \int \frac{d^3k}{(2\pi \hbar)^3}\mathbf{k} \times \sum_{s,s'}\langle s|\pmb{\bm{\sigma}}|s'\rangle \ n(k)
 \end{split}
\end{equation}

The different terms can be grouped together in simpler notation:
\begin{equation}
 \begin{split}
  \varepsilon &= \sum_{t=n,p}\frac{\hbar^2}{2M_q}\tau_q+\frac{1}{4}t_0\Big[(2+x_0)\rho^2-(1+2x_0)\sum_{q=n,p}\rho^2_q\Big]\\
  &+\frac{1}{8}\Big[ a (\rho \tau+\frac{3}{4}(\triangledown \rho)^2)+2b\sum_{q=n,p}(\rho_q \tau_q+\frac{3}{4}(\triangledown \rho_q)^2)\Big]\\
  &+\frac{1}{24}t_3\rho^{\alpha}\Big[(2+x_3)\rho^2-(1+2x_3)\sum_{q=n,p}\rho_q^2\Big]\\
  &+\frac{1}{8}t_4\Big[ (2+x_4)(\rho \tau+\frac{3}{4}(\triangledown \rho)^2)-(1+2x_4)\sum_{q=n,p}(\rho_q \tau_q+\frac{3}{4}(\triangledown \rho_q)^2)\Big]\rho^{\beta}\\
 &+\frac{\beta}{16}t_4\Big[(2+x_4)\rho(\triangledown \rho)^2-(1+2x_4)\mathbf{\triangledown}\rho \cdot \sum_{q=n,p}\rho_q\mathbf{\triangledown}\rho_q\Big]\rho^{\beta-1}\\
 &+\frac{1}{8}t_5\Big[ (2+x_5)(\rho \tau+\frac{3}{4}(\triangledown \rho)^2)+(1+2x_5)\sum_{q=n,p}(\rho_q \tau_q-\frac{1}{4}(\triangledown \rho_q)^2)\Big]\rho^{\gamma}\\
 &-\frac{1}{16}(t_1x_1+t_2x_2)J^2+\frac{1}{16}(t_1-t_2)\sum_{q=n,p}J^2_q\\
 &-\frac{1}{16}(t_4 x_4\rho^{\beta}+t_5x_5\rho^{\gamma})J^2+\frac{1}{16}(t_4\rho^{\beta}-t_5\rho^{\beta})\sum_{q=n,p}J^2_q\\
 &+\frac{1}{2} W_0 (\mathbf{J}\cdot \mathbf{\triangledown}\rho+\sum_{q=n,p}\mathbf{J}_q\cdot \mathbf{\triangledown}\rho_q)
 \end{split}
\end{equation}
where, $a = t_1(x_1+2)+t_2(x_2+2), \ b= \frac{1}{2}[t_2(2 x_2+1)-t_1(2x_1+1)]$.

In uniform matter $\triangledown \rho =0$:
\begin{equation}
 \begin{split}
  \varepsilon &= \sum_{t=n,p}\frac{\hbar^2}{2M_q}\tau_q+\frac{1}{4}t_0\Big[(2+x_0)\rho^2-(1+2x_0)\sum_{q=n,p}\rho^2_q\Big]\\
  &+\frac{1}{8}\Big[ a \rho \tau+2b\sum_{q=n,p}\rho_q \tau_q\Big]\\
  &+\frac{1}{24}t_3\rho^{\alpha}\Big[(2+x_3)\rho^2-(1+2x_3)\sum_{q=n,p}\rho_q^2\Big]\\
  &+\frac{1}{8}t_4\Big[ (2+x_4)\rho \tau-(1+2x_4)\sum_{q=n,p}\rho_q \tau_q\Big]\rho^{\beta}\\
 &+\frac{1}{8}t_5\Big[ (2+x_5)\rho \tau+(1+2x_5)\sum_{q=n,p}\rho_q \tau_q\Big]\rho^{\gamma}\\
 &-\frac{1}{16}(t_1x_1+t_2x_2)J^2+\frac{1}{16}(t_1-t_2)\sum_{q=n,p}J^2_q\\
 &-\frac{1}{16}(t_4 x_4\rho^{\beta}+t_5x_5\rho^{\gamma})J^2+\frac{1}{16}(t_4\rho^{\beta}-t_5\rho^{\beta})\sum_{q=n,p}J^2_q\\
 \end{split}
\end{equation}
In unpolarized matter, $\mathbf{J}=0$:
\begin{equation}
 \begin{split}
  \varepsilon &= \sum_{t=n,p}\frac{\hbar^2}{2M_q}\tau_q+\frac{1}{4}t_0\Big[(2+x_0)\rho^2-(1+2x_0)\sum_{q=n,p}\rho^2_q\Big]\\
  &+\frac{1}{8}\Big[ a \rho \tau+2b\sum_{q=n,p}\rho_q \tau_q\Big]\\
  &+\frac{1}{24}t_3\rho^{\alpha}\Big[(2+x_3)\rho^2-(1+2x_3)\sum_{q=n,p}\rho_q^2\Big]\\
  &+\frac{1}{8}t_4\Big[ (2+x_4)\rho \tau-(1+2x_4)\sum_{q=n,p}\rho_q \tau_q\Big]\rho^{\beta}\\
 &+\frac{1}{8}t_5\Big[ (2+x_5)\rho \tau+(1+2x_5)\sum_{q=n,p}\rho_q \tau_q\Big]\rho^{\gamma}\\
 \end{split}
 \label{eq:SkyrmeDFT}
\end{equation}
Energy per bayon, $\mathcal{E}\equiv \varepsilon/rho$:
\begin{equation}
 \begin{split}
  \mathcal{E}&= \sum_{t=n,p}\frac{\hbar^2}{2M_q}\frac{\tau_q}{\rho}+\frac{1}{4}t_0\Big[(2+x_0)\rho-(1+2x_0)\sum_{q=n,p}\frac{\rho^2_q}{\rho}\Big]\\
  &+\frac{1}{8}\Big[ a \tau+2b\sum_{q=n,p}\frac{\rho_q \tau_q}{\rho}\Big]\\
  &+\frac{1}{24}t_3\rho^{\alpha}\Big[(2+x_3)\rho-(1+2x_3)\sum_{q=n,p}\frac{\rho_q^2}{\rho}\Big]\\
  &+\frac{1}{8}t_4\Big[ (2+x_4) \tau-(1+2x_4)\sum_{q=n,p}\frac{\rho_q \tau_q}{\rho}\Big]\rho^{\beta}\\
 &+\frac{1}{8}t_5\Big[ (2+x_5) \tau+(1+2x_5)\sum_{q=n,p}\frac{\rho_q \tau_q}{\rho}\Big]\rho^{\gamma}\\
 \end{split}
\end{equation}
In terms of proton fraction, $y = \frac{\rho_p}{\rho_p+\rho_n}$:
\begin{equation}
 \begin{split}
  \mathcal{E}&= \sum_{t=n,p}\frac{\hbar^2}{2M_q}\frac{\tau_q}{\rho}+\frac{1}{4}t_0\Big[(2+x_0)-(1+2x_0)[y^2+(1-y)^2]\Big]\rho\\
  &+\frac{1}{8}\Big[ a \tau+2b[y \tau_p + (1-y)\tau_n]\Big]\\
  &+\frac{1}{24}t_3\rho^{\alpha+1}\Big[(2+x_3)-(1+2x_3)[y^2+(1-y)^2]\Big]\\
  &+\frac{1}{8}t_4\Big[ (2+x_4) \tau-(1+2x_4)[y \tau_p + (1-y)\tau_n]\Big]\rho^{\beta}\\
 &+\frac{1}{8}t_5\Big[ (2+x_5) \tau+(1+2x_5)[y \tau_p + (1-y)\tau_n]\Big]\rho^{\gamma}\\
 \label{eq:SkyrmeEDF}
 \end{split}
\end{equation}
\section{Single particle properties}
From the energy density the single particle spectrum can be derived. By performing functional variation of the energy density with repsect to the single 
particle wavefunction, a modified Schrodinger equation can be derived:
\begin{equation}
 \begin{split}
  \delta \varepsilon_i =&  \big[\frac{\delta \varepsilon_i}{\delta \tau_i} +  \frac{\delta \varepsilon_i}{\delta \rho_i}] \delta \phi_i = \epsilon_i \delta \phi_i  
 \end{split}
\end{equation}
Since the Skyrme potential is at mostly quadratic in momenta with nonlinaer density dependence, its effect is 
exactly included by effective mass and mean field shift(residual interaction), both density dependent:
\begin{equation}
 \begin{split}
 \epsilon_i(k) =& \frac{\hbar^2k^2}{2 M^{*}_i}+U_i\\
  \frac{\hbar^2}{2 M^{*}_q} \equiv& \frac{\partial \varepsilon}{\partial \tau_q}\\
  U_i\equiv&\frac{\partial \varepsilon}{\partial \rho_i}\\
 \end{split}
\end{equation}
From eq.~\ref{eq:SkyrmeDFT} the effective baryon masses:
\begin{equation}
 \begin{split}
  M^{*}_p/M =&  \Big\{1 + \frac{M\ \rho}{4 \hbar^2}   \big[a + 2\ y\ b  + t_4 [(2 + x_4) - (1 +2 x_4)\ y ] \rho^\beta +  t_5 [2 + x_5 + (1 + 2 x_5)\ y] \rho^{\gamma}\big]\Big\}^{-1}\\
   M^{*}_n/M =&   \Big\{1 + \frac{M\ \rho}{4 \hbar^2}   \big[a + 2\ (1-y)\ b  + t_4 [(2 + x_4) - (1 +2 x_4)\ (1-y) ] \rho^\beta +  t_5 [2 + x_5 + (1 + 2 x_5)\ (1-y)] \rho^{\gamma}\big]\Big\}^{-1}
 \end{split}
\end{equation}
and the residual potentials:
\begin{equation}
 \begin{split}
  U_p=&\frac{ 1}{8}(2 b\ \tau_p  +a\ \tau) +\frac{1}{2}t_0 [(2+x_0)-(1+2 x_0)\ y]\rho\\
  &+ \frac{1}{24}t_3\Big[4 + \alpha - 2 y (1 - (1 - y) \alpha) +  x_3 (1 - 2 y) [2 - (1 - 2 y) \alpha]\Big]\rho^{\alpha+1}\\
  U_n=&\frac{ 1}{8}(2 b\ \tau_n  +a\ \tau) +\frac{1}{2}t_0 [(1 - x_0) + (1 + 2 x_0) y] \rho \\
  &+ \frac{1}{24}t_3\Big[2 + \alpha + 2 y (1 + \alpha - y \alpha) - x_3 (1 - 2 y) [2 + (1 - 2 y) \alpha]\Big]\rho^{\alpha+1}
 \end{split}
\end{equation}

\section{T=0 DFT}
At $T=0$, there are simple relation that can be drawn between the 2 integrations in Fourier space since the occuppation number is a step function:
\begin{equation}
 \begin{split}
  \rho_q &= 2\int \frac{d^3k}{(2\pi \hbar)^3} \theta(k_{F,q}-k)=\frac{k_{F,q}^3}{3 \pi^2 \hbar^3}\\
  \tau_q &=2\int \frac{d^3k}{(2\pi \hbar)^3} (k/\hbar)^2 \theta(k_{F,q}-k)=\frac{k_{F,q}^5}{5 \pi^2 \hbar^5} \rightarrow\\
  \tau &= \frac{3}{5} (3 \pi^2)^{2/3} \rho_q^{5/3}, H_n(y) = 2^{n-1}[y^n+(1-y)^n], y =\rho_p/\rho\\
  \tau &= \tau_p+\tau_n =\frac{3}{5}(\frac{3 \pi^2}{2})^{2/3} H_{5/3}(y) \rho^{5/3}
 \end{split}
\end{equation}
So,
\begin{equation}
 \begin{split}
  \mathcal{E}_{0} &=\frac{3\hbar^2}{10M}(\frac{3\pi^2}{2})^{2/3}\rho^{2/3} H_{5/3}(y)+\frac{1}{8}t_0\rho \Big[2(2+x_0)-(1+2x_0)H_2(y)\Big]\\
  &+\frac{3}{40}(\frac{3 \pi^2}{2})^{2/3}\Big[ a H_{5/3}(y) +b H_{8/5}(y) \Big]\rho^{5/3}\\
  &+\frac{1}{48}t_3\rho^{\alpha+1}\Big[2(2+x_3)-(1+2x_3)H_2(y)\Big]\\
  &+\frac{3}{40}(\frac{3 \pi^2}{2})^{2/3}t_4\rho^{\beta+5/3}\Big[ (2+x_4)H_{5/3}(y)-(\frac{1}{2}+x_4)H_{8/3}(y)\Big]\\
 &+\frac{3}{40}(\frac{3\pi^2}{2})^{2/3}t_5\rho^{\gamma+5/3}\Big[ (2+x_5)H_{5/3}(y)+(\frac{1}{2}+x_5)H_{8/3}(y) \Big]\\
 \end{split}
\end{equation}
with $H_n(y)= 2^{n-1}[y^n+(1-y)^n]$. A common choice is to set $M = 1/2(M_n+M_p)$, or use the individual value for each species.
In compact notation,
\begin{equation}
 \begin{split}
  \mathcal{E}_{0} &= C(y)\rho^{2/3}+ A(y) \rho + B(y) \rho^{\alpha+1} + D(y) \rho^{5/3} + G(y) \rho^{\beta+5/3} + K(y)\rho^{\gamma+5/3}
 \end{split}
\end{equation}
By comparing the 2 expressions, the following relations can be easily deduced:
\begin{equation}
 \begin{split}
  C(y)=&\frac{3\hbar^2}{10M_y}(\frac{3\pi^2}{2})^{2/3} H_{5/3}(y)\\
  A(y)=&\frac{1}{8}t_0 \Big[2(2+x_0)-(1+2x_0)H_2(y)\Big]\\
  B(y)=&\frac{1}{48}t_3\Big[2(2+x_3)-(1+2x_3)H_2(y)\Big]\\
  D(y)=&\frac{3}{40}(\frac{3 \pi^2}{2})^{2/3}\Big[ a H_{5/3}(y) +b H_{8/5}(y) \Big]\\
  G(y)=&\frac{3}{40}(\frac{3 \pi^2}{2})^{2/3}t_4\Big[ (2+x_4)H_{5/3}(y)-(\frac{1}{2}+x_4)H_{8/3}(y)\Big]\\
  K(y)=&\frac{3}{40}(\frac{3\pi^2}{2})^{2/3}t_5\Big[ (2+x_5)H_{5/3}(y)+(\frac{1}{2}+x_5)H_{8/3}(y) \Big]
 \end{split}
\end{equation}
For pure neutron matter and symmetric matter the kinetic coefficient is 
\begin{equation}
 \begin{split}
C_n =& C(0) = \frac{3\hbar^2}{10M_n}(\frac{3\pi^2}{2})^{2/3}H_{5/3}(0) = 118.995\ \text{Mev}\ \text{fm}^{2} \approx 119\ \text{Mev}\ \text{fm}^{2}\\
C_{sym}=& C(1/2) =  \frac{3\hbar^2}{5(M_n+M_p)}(\frac{3\pi^2}{2})^{2/3}H_{5/3}(1/2) = 75.0139 \ \text{Mev}\ \text{fm}^{2} \approx75\ \text{Mev}\ \text{fm}^{2} 
 \end{split}
\end{equation}
The effective mass is due to the terms dependent on kinetic energy:
\begin{equation}
 \begin{split}
  \tau^{T=0}(\rho,y) \equiv& \frac{3\hbar^2}{10M^{*}}(\frac{3\pi^2}{2})^{2/3}\rho^{2/3} H_{5/3}(y)\\
  =&\frac{3\hbar^2}{10M}(\frac{3\pi^2}{2})^{2/3}\rho^{2/3} H_{5/3}(y) +\frac{3}{40}(\frac{3 \pi^2}{2})^{2/3}\Big[ a H_{5/3}(y) +b H_{8/5}(y) \Big]\rho^{5/3}\\
  &+\frac{3}{40}(\frac{3 \pi^2}{2})^{2/3}t_4\rho^{\beta+5/3}\Big[ (2+x_4)H_{5/3}(y)-(\frac{1}{2}+x_4)H_{8/3}(y)\Big]\\
 &+\frac{3}{40}(\frac{3\pi^2}{2})^{2/3}t_5\rho^{\gamma+5/3}\Big[ (2+x_5)H_{5/3}(y)+(\frac{1}{2}+x_5)H_{8/3}(y) \Big]\\
 \equiv& C(y)\rho^{2/3} \Big[1 +( D(y) \rho + G(y) \rho^{\beta+1} + K(y)\rho^{\gamma+1})/C(y)\Big]
 \end{split}
\end{equation}
Thus,
\begin{equation}
 \begin{split}
  M^{*}/M= \{&1+\frac{M}{4 H_{5/3}(y) \hbar^2}\Big[ \rho[a H_{5/3}(y) +b H_{8/5}(y)]+\rho^{\beta+1} t_4[ (2+x_4)H_{5/3}(y)-(\frac{1}{2}+x_4)H_{8/3}(y)]\\
  &+\rho^{\gamma+1} t_5[ (2+x_5)H_{5/3}(y)-(\frac{1}{2}+x_5)H_{8/3}(y)]\Big]\}^{-1}\\
  =&\frac{C(y)}{C(y)+ D(y) \rho + G(y) \rho^{\beta+1} + K(y)\rho^{\gamma+1}}
 \end{split}
\end{equation}
Also, the thermodynamic pressure:
\begin{equation}
 \begin{split}
  \mathcal{P}^{T=0}(\rho,y) &= \rho^2 \frac{\partial \mathcal{E}^{T=0}}{\partial \rho}\\
  &=\frac{2}{3}C(y)\rho^{5/3}+A(y)\rho^2+(\alpha+1)B(y)\rho^{\alpha+1}+\frac{5}{3}D(y)\rho^{8/3}+\beta G(y) \rho^{\beta+8/3}+ \gamma K(y) \rho^{\gamma+8/3}
 \end{split}
\end{equation}
\section{Lattimer - Schwesty Notation}
The standard parametrization:
\begin{equation}
 \begin{split}
  \varepsilon =& \frac{{\hbar}^2 {\tau_n}}{2 {M_n}}+\frac{{\hbar}^2 {\tau_p}}{2 t{M_p}}+\frac{1}{8} [a ({\rho_n}+{\rho_p}) ({\tau_n}+{\tau_p})+2 b ({\rho_n} {\tau_n}+{\rho_p} {\tau_p})]+\frac{1}{4} t_0 [(2 + x_0) (\rho_n + \rho_p)^2 - (1 +  2 x_0) (\rho_n^2 + \rho_p^2)]\\
  &+\frac{1}{24} t_3 (\rho_n + \rho_p)^\alpha [(2 +  x_3) (\rho_n + \rho_p)^2 - (1 + 2 x_3) (\rho_n^2 + \rho_p^2)]
 \end{split}
\end{equation}
\begin{minipage}[t]{0.5\textwidth}
  \begin{equation}
 \begin{split}
  a=& \frac{4 \hbar^2}{M}(F+G)\\
  b=&-\frac{4 \hbar^2}{M} G\\
  t_0=& \frac{8}{3}(A+B)\\
  x_0=&-\frac{1}{2}\frac{A-2 B}{A+B}\\
  t_3=&16(C+D)\\
  x_3=&-\frac{1}{2}\frac{C-D}{C+D}\\
  \alpha =& \delta -1
 \end{split}
\end{equation}
\end{minipage}
\begin{minipage}[t]{0.5\textwidth}
  \begin{equation}
 \begin{split}
  a=& \frac{4 \hbar^2}{M}(F+G)\\
  b=&-\frac{4 \hbar^2}{M} G\\
  t_0=& \frac{8}{3}(A+B)\\
  x_0=&-\frac{1}{2}\frac{A-2 B}{A+B}\\
  t_3=&16(C+D)\\
  x_3=&-\frac{1}{2}\frac{C-D}{C+D}\\
  \alpha =& \delta -1
 \end{split}
\end{equation}
\end{minipage}

\begin{equation}
 \begin{split}
  \varepsilon =& 4 B \rho_n \rho_p + A (\rho_n + \rho_p)^2 + (\rho_n + \rho_p)^{\delta-1} [4 D \rho_n \rho_p + C (\rho_n + \rho_p)^2]\\
  &+ \frac{5(\frac{2}{3})^{2/3}}{3\pi^{4/3}}\alpha_S [M(\frac{\tau_n}{M_n}+\frac{\tau_p}{M_p}) +F (\rho_n+\rho_p)(\tau_n+\tau_p)-G(\rho_n-\rho_p)(\tau_n-\tau_p)]
 \end{split}
\end{equation}

The standard parametrization for $T = 0,\ (M_n,M_p) \rightarrow M = \frac{1}{2}(M_n+M_p)$:
\begin{equation}
 \begin{split}
  \varepsilon &=\frac{3\hbar^2}{10M}(\frac{3\pi^2}{2})^{2/3}\rho^{5/3} H_{5/3}(y)+\frac{1}{8}t_0\rho^2 \Big[2(2+x_0)-(1+2x_0)H_2(y)\Big]\\
  &+\frac{3}{40}(\frac{3 \pi^2}{2})^{2/3}\Big[ a H_{5/3}(y) +b H_{8/5}(y) \Big]\rho^{8/3}\\
  &+\frac{1}{48}t_3\rho^{\alpha+2}\Big[2(2+x_3)-(1+2x_3)H_2(y)\Big]\\
 \end{split}
\end{equation}
By comparing the expression in these notes with the ones from  Lattimer parametrization, the energy density is:
\begin{equation}
 \begin{split}
\varepsilon = \alpha_S \rho^{5/3}H_{5/3}(y)+[A+B(2-H_{2}(y))]\rho^2+ [C+D(2-H_{2}(y))]\rho{\delta}+\alpha_S \rho^{5/3}[(F+G)H_{5/3}-GH_{8/3}(y)]
 \end{split}
\end{equation}
where, $\alpha_S=\frac{3\hbar^2}{10 M}(\frac{3}{2}\pi^2)^{2/3}$
\subsection{Skyrme parametrization from Saturation Observables}
Given the following set of physical observables,
\begin{equation}
 \begin{split}
  E_0=&\varepsilon\Big|_{\rho_0,y=1/2}=(A+B) \rho_0^2+(C+D) \rho_0^{\delta+1}+\alpha_S \rho_0^{5/3} (1+F \rho_0)\\
  P=&\rho^2\frac{d(\varepsilon/\rho)}{d\rho}\Big|_{\rho_0,y=1/2}=\frac{2}{3} \alpha_S \rho_0^{5/3}+(A+B) \rho_0^2+\frac{5}{3} F \alpha_S \rho_0^{8/3}+(C+D) \delta  \rho_0^{1+\delta }=0\\
  (M^{*}/M)=&\frac{d\varepsilon}{d\tau}\Big|_{\rho_0,y=1/2}=(1+F \rho_0)^{-1}\\
  K_m=&9\rho^2\frac{d^2(\varepsilon/ \rho)}{d\rho^2}\Big|_{\rho_0,y=1/2}=-2 \alpha_S \rho_0^{2/3} + 10 F \alpha_S \rho_0^{5/3} + 9 (C + D) (\delta-1) \delta \rho_0^\delta\\
  S =& \frac{1}{8}\frac{d^2(\varepsilon/ \rho)}{dy^2}\Big|_{\rho_0,y=1/2}=\frac{5}{9} \alpha_S \rho_0^{2/3} - B \rho_0 + \frac{5}{9} (F - 3 G) \alpha_S \rho_0^{5/3} - D \rho_0^\delta\\
  L=& 3\rho \frac{d S}{\rho}\Big|_{\rho_0,y=1/2}=\frac{10}{9} \alpha_S \rho_0^{2/3} - 3 B \rho_0 + \frac{25}{9} (F - 3 G) \alpha_S \rho_0^{5/3} - 3 D \delta \rho_0^\delta\\
  K_s=&9 \rho^2\frac{d^2S}{d\rho^2}\Big|_{\rho_0,y=1/2}=-\frac{10}{9} \alpha_S \rho_0^{2/3} + \frac{50}{9} (F - 3 G) \alpha_S \rho_0^{5/3} - 9 D (\delta-1) \delta \rho_0^\delta
 \end{split}
\end{equation}

the skyrme parameters can be found as follows,

\begin{equation}
 \begin{split}
  F=&\frac{(M^{*}/M)^{-1}-1}{\rho_0}\\
  \delta=&\frac{K_m+2 \rho_0^{2/3}(1-5 F\rho_0)\alpha_S}{3 \rho_0^{2/3}(1-2 F\rho_0)\alpha_S-9 E_0}\\
  G=&\frac{9 K_S-27 (L-3 S) \delta +5 \rho_0^{2/3} \alpha_S [2-3 \delta +2 F \rho_0 (3 \delta-5)]}{30 \rho_0^{5/3} \alpha_S (3 \delta-5)}\\
  D=&\frac{5 (3 L-9 S+\rho_0^{2/3} \alpha_S)-3 K_S}{9(5-8 \delta +3 \delta ^2)\rho_0^{\delta }}\\
  C=&\frac{ \rho_0^{2/3} (1-2 F \rho_0) \alpha_S-3 E_0}{3 (\delta-1)\rho_0^{\delta }}-D\\
  B=&\frac{ L (6+9 \delta )+5 ( \rho_0^{2/3} \alpha_S (3 \delta -2)-9 S \delta)-3 K_S}{18 \rho_0 (\delta-1)}\\
  A=&-[\frac{2}{3} \alpha_S \rho ^{-1/3}+ B  +\frac{5}{3}F \alpha_S \rho ^{2/3}+ (C + D) \delta  \rho ^{\delta-1 }]
 \end{split}
\end{equation}



\section{Finite Temperature DFT}
In order to obtain the relationship between density and chemical potential, the following set of coupled equations need to be solved self-consistenly:
 \begin{equation}
  \begin{split}
  f_k&=\bigg [1+ e^{\big(\frac{k^2}{2M^{*}}+U-\mu \big)/T}\bigg ]^{-1}\\
  \rho &= \sum_{s,is}\int \frac{d^3k}{(2\pi)^3}f_k\\
  \tau &= \sum_{s,is}\int \frac{d^3k}{(2\pi)^3}k^2 f_k\\
  E &\equiv E(\rho,\tau)\\
  \end{split}
  \label{eq:sceq1}
  \end{equation}
  where, a sum over all discrete quantum numbers is performed (spin and isospin).
  And from the energy density functional, the mean field parameters can be derived,
  \begin{equation}
   \begin{split}
  M^{*}&=\frac{1}{2}(\frac{\delta E}{\delta \tau})^{-1}\\
  U &= \frac{\delta E}{\delta \rho}
   \end{split}
   \label{eq:sceq2}
  \end{equation}
The chemical potential can be found 
by inverting the expression for the density.

\section{Thermodynamic Potentials}
Since the effect of phenomenological mean field models can be incorporated into ($M^{*},U$) which are density dependent for Skyrme, and also
temperature dependent for RMF, the thermodynamic properties of assymetric matter at finite temperature can be expressed by fermi integrals of
`modified' non-interacting fermi gases.
The single particle spectrum and dustribution fucntion:
\begin{equation}
 \begin{split}
 \xi_i =& \frac{k^2}{2 M_i^{*}}+U_i\\
  F_i=& \Big\{ \exp \big[\frac{\xi_i-\mu_i}{T} \big] + 1\Big \}^{-1}= \Big\{ \exp \big[\frac{\frac{k^2}{2 M^{*}_i}-\eta_i}{T} \big] + 1\Big \}^{-1},\ \overline{F}=1 -F\\
  \end{split}
  \end{equation}
  The density and kinetic density:
  \begin{equation}
   \begin{split}
  \rho_i=&\int_0^{\infty} \frac{dk}{\pi^2}k^2 F_i\\
  \tau_i=&\int_0^{\infty} \frac{dk}{\pi^2}k^4 F_i\\
   \end{split}
  \end{equation}
  The entropy density can be calcualted fro mthe distribution function:
  \begin{equation}
   \begin{split}
  S/V=&-\int_0^{\infty}\frac{dk}{\pi^2}\ k^2 \big [ F_i\ln(F_i) + (1-F_i)\ln (1-F_i)\big]\\
  =&-\int_0^{\infty}\frac{dk}{\pi^2}\ k^2 F_i \ln(\frac{F_i}{1-F_i})-\int_0^{\infty}\frac{dk}{\pi^2}\ k^2 \ln (1-F_i)\\
  =&-\int_0^{\infty}\frac{dk}{\pi^2}\ k^2 F_i \ln(\exp \big[\frac{-\xi_i+\mu_i}{T} \big])-[\frac{k^3}{3 \pi^2}\ln (1-F_i)]
  \Big|_0^{\infty}+\int_0^{\infty}\frac{dk}{\pi^2}\ \frac{k^3}{3} \frac{k}{M_i^{*}}\frac{\exp \big[\frac{\xi_i-\mu_i}{T} \big]}{1-F_i}\\ 
  =&\big[\frac{1}{2 M_i^{*} T}+\frac{1}{3 M_i^{*}T}\big]\int_0^{\infty}\frac{dk}{\pi^2}\ k^4 F_i+\frac{U_i-\mu_i}{T}\ \int_0^{\infty}\frac{dk}{\pi^2}\ k^2 F_i\\
  =&\frac{1}{T}[\frac{5\tau_i}{6M_i^{*}}+(U_i-\mu_i)\rho]
   \end{split}
  \end{equation}
From the first law of thermodynamics:
  \begin{equation}
  \begin{split}
  E_i =& TS_i+\mu_i N_i-P_iV\\
  P_i=&\frac{T S_i + \mu_i N_i - E_i}{V}\\
    =&\frac{5 \tau_i}{6 M_i^{*}}+U_i \rho_i-\frac{E_i}{V}
  \end{split}
  \end{equation}
\section{Thermodynamic Derivatives}
Let,
\begin{equation}
 \begin{split}
  \alpha_1=&\frac{\hbar^2}{2M}(F-G), \alpha_2 =\frac{\hbar^2}{2M} (G+F)
  \end{split}
\end{equation}
Then,
\begin{equation}
 \begin{split}
  \frac{\hbar^2}{2 M^{*}_i}=&\frac{\hbar^2}{2 M}+\alpha_1 \rho_i+\alpha_2 \rho_{-i}\rightarrow\\
  M^{*}_i=&M \big[1+F \rho- \text{sgn}(i) G (\rho_n-\rho_p)\big]^{-1}\\
  U_i=&\alpha_1 \tau_i+\alpha_2 \tau_{-i}+2A \rho + 4 B \rho_{-i}+ C(1+\delta)\rho^{\delta}+4 D \rho_{-i}(\rho_{-i}+\delta \rho_i)\rho^{\delta-2}\\
  \rho=&\rho_n+\rho_p
 \end{split}
\end{equation}
where, $i$ denotes the isospin value. The set of independent parameters which is used for the $1^{st}$ part of thsi section is $(\rho_n,\rho_p,T)$.
From epxressions above, the partial derivatives can be found:
\begin{equation}
 \begin{split}
  \partial_{\rho_i}M^{*}_r=&-\frac{{M_r^{*}}^2}{M}\big[F+G(1-2\delta_{ir})\big]\\
  =&-M \big[F+G(1-2\delta_{ir})\big]\big[1+F \rho- \text{sgn}(i) G (\rho_n-\rho_p)\big]^{-2}\\
  \partial_{T}M^{*}_r =& 0\\
  \partial_{\rho_i}U_r=& \alpha_1 \partial_{\rho_i}\tau_r+\alpha_2 \partial_{\rho_i}\tau_{-r} + 2 A + 4 B (1-\delta_{ir}) + C \delta (1+\delta)\rho^{\delta-1}\\
  &+4 D \rho^{\delta-3} \big[(\delta-1)\rho_{-r}(2\rho_{-r}+\delta \rho_r)\delta_{ir}+(1-\delta_{-ir})[\delta (\rho_{-r}^2+\rho_{r}^2)+\rho_{-r}\rho_{r}(2+\delta(\delta-1))]\big]\\
  \partial_{T}U_r =& \alpha_1 \partial_{T}\tau_r+\alpha_2 \partial_{T}\tau_{-r}
 \end{split}
\end{equation}
The number density and kinetic density can be expressed in terms of the general fermi integration:
\begin{equation}
 \begin{split}
  F_n(\eta)=&\int_0^{\infty}\frac{u^n}{e^{u-\eta}+1}du,\ \eta_i=(\mu_i-U_i)/T\\
  \tau_r=&\frac{1}{2\pi^2}(\frac{2M^{*}_rT}{\hbar^2})^{5/2}F_{3/2}(\eta_r)\\
  \rho_r=&\frac{1}{2\pi^2}(\frac{2M_r^{*}T}{\hbar^2})^{3/2}F_{1/2}(\eta_r)\\
 \end{split}
\end{equation}
By inverting the expression for the density:
\begin{equation}
 \begin{split}
  \eta_r= F^{-1}_{1/2}\big[2 \pi^2 \rho_r(\frac{\hbar^2}{2 M_r^{*}T})^{3/2}\big]\equiv F^{-1}_{1/2}\big[\eta^{(-1)}_r\big]\leftrightarrow\\
  \eta_r= F^{-1}_{1/2}\big[\eta^{(-1)}_r\big],\ \eta^{(-1)}_r=2 \pi^2 \rho_r(\frac{\hbar^2}{2 M_r^{*}T})^{3/2} = F_{1/2}\big[\eta_r\big]
 \end{split}
\end{equation}
And, 
\begin{equation}
 \begin{split}
  \partial_{\rho_i}\eta^{(-1)}_r=&\big[\frac{\delta_{ir}}{\rho_r}-\frac{3}{2}\frac{\partial_{\rho_i}M_r^{*}}{M_r^{*}}\big] \eta^{(-1)}_r\\
  =&\Big[\frac{\delta_{ir}}{\rho_r}+\frac{3}{2}\frac{M_r^{*}}{M}\big[F+G(1-2\delta_{ir})\big]\Big] \\
  =&\frac{F_{1/2}(\eta_r)}{\rho_r}\Big[\delta_{ir}+\frac{3\rho_r}{2}\frac{M_r^{*}}{M}\big[F+G(1-2\delta_{ir})\big]\Big]\\
  \partial_{T}\eta^{(-1)}_r=&-\frac{3}{2}\frac{\eta_r^{(-1)}}{T}\\
  =&-\frac{3}{2}\frac{F_{1/2}(\eta_r)}{T}
 \end{split}
\end{equation}
Also, $\partial_{\eta}F_n(\eta)=nF_{n-1}(\eta) \leftrightarrow \partial_{u}F^{-1}_n(u)=\frac{1}{nF_{n-1}(u)}$:
\begin{equation}
 \begin{split}
  \partial_{\rho_i}\eta_r=&\frac{\partial_{\rho_i}\eta_r^{(-1)}}{\partial_{\eta_r}F_{1/2}(\eta_r)}\\
  =&\frac{2F_{1/2}(\eta_r)}{F_{-1/2}(\eta_r)\rho_r}\Big[\delta_{ir}+\frac{3\rho_r}{2}\frac{M_r^{*}}{M}\big[F+G(1-2\delta_{ir})\big]\Big]\\
  =&\frac{G_r}{\rho_r}\Big[\delta_{ir}+\frac{3\rho_r}{2}\frac{M_r^{*}}{M}\big[F+G(1-2\delta_{ir})\big]\Big]\\
  \partial_{T}\eta_r=&\frac{\partial_{T}\eta_r^{(-1)}}{\partial_{\eta_r}F_{1/2}(\eta_r)}\\
  =&-\frac{3}{2T}\frac{2F_{1/2}(\eta_r)}{F_{-1/2}(\eta_r)}\\
  =&-\frac{3}{2}\frac{G_r}{T}
 \end{split}
\end{equation}
where, $G_r = \frac{2 F_{1/2}(\eta_r)}{F_{-1/2}(\eta_r)}$.
Thus,
\begin{equation}
 \begin{split}
 \partial_{\rho_i}\tau_r=&\frac{5}{2}\tau_r\frac{\partial_{\rho_i}M_r^{*}}{ M_r^{*}}+\frac{3}{2}(\frac{2M_r^{*}T}{\hbar^2})\rho_r\partial_{\rho_i}\eta_r\\
 =&-\frac{5}{2}\tau_r\frac{M_r^{*}}{M}\big[F+G(1-2\delta_{ir})\big]+\frac{3}{2}(\frac{2M_r^{*}T}{\hbar^2})G_r\Big[\delta_{ir}+\frac{3\rho_r}{2}\frac{M_r^{*}}{M}\big[F+G(1-2\delta_{ir})\big]\Big]\\
  =&3 T G_r\frac{M_r^{*}}{\hbar^2} \delta_{ir} +\frac{1}{2}\frac{M_r^{*}}{M}(\frac{9 M_r^{*}}{\hbar^2} T \rho_r G_r - 5 \tau_r)\big[F+G(1-2\delta_{ir})\big]\\
  \partial_{T}\tau_r=&\frac{5}{4\pi^2 T}(\frac{2 M^{*}_rT}{\hbar^2})^{5/2} F_{3/2}(\eta_r)+\frac{1}{2\pi^2}(\frac{2 M_r^{*}T}{\hbar^2})^{5/2}\partial_{\eta_r}F_{3/2}(\eta_r)\partial_{T}\eta_r\\
  =&\frac{5}{2T}\tau_r-\frac{9}{8\pi^2 T}(\frac{2 M^{*}_rT}{\hbar^2})^{5/2} F_{1/2}(\eta_r)G_r\\
  =&\frac{5}{2T}\tau_r-\frac{9}{2}\frac{M_r^{*}}{\hbar^2}G_r\rho_r\\
 \end{split}
\end{equation}

Now, we are ready to find the partial derivatives of the chemical potential,
\begin{equation}
 \begin{split}
  \partial_{\rho_i}\mu_r =& T\partial_{\rho_i}\eta_r + \partial_{\rho_i} U_r\\
  =&\frac{TG_r}{\rho_r}\Big[\delta_{ir}+\frac{3 \rho_r M_r^{*}}{\hbar^2}\big[\alpha_2+(\alpha_1-\alpha_2)\delta_{ir}\big]\Big] + \partial_{\rho_i} U_r\\
  =&\frac{TG_r}{\rho_r}\Big[\delta_{ir}+\frac{3}{2}\rho_r\big[F+G(1-2\delta_{ir})\big]\Big] + \partial_{\rho_i} U_r\\
  \partial_{T}\mu_r=& \eta_r + T \partial_{T}\eta_r +\partial_{T}U_r\\
  =&\eta_r - \frac{3}{2}G_r+\alpha_1 \partial_{T}\tau_r+\alpha_2 \partial_{T}\tau_{-r}\\
  =&\eta_r - \frac{3}{2}G_r+\frac{5}{2T}(\alpha_1\tau_r+\alpha_2\tau_{-r})- \frac{9}{2} (\alpha_1 \frac{M_r^{*}}{\hbar^2}\rho_r G_r+\alpha_2 \frac{M_{-r}^{*}}{\hbar^2}\rho_{-r} G_{-r})\\
  =&\eta_r - \frac{3}{2}G_r+\sum_i (\frac{5\tau_i}{2T}-\frac{9M_i^{*}}{2 \hbar^2}\rho_i G_i) \big[\alpha_2+(\alpha_1-\alpha_2)\big]\\
  =&\eta_r - \frac{3}{2}G_r+\sum_i (\frac{5\tau_i \hbar^2}{4TM_i^{*}}-\frac{9}{4}\rho_iG_i) \big[F+G(1-2\delta_{ir})\big]
 \end{split}
\end{equation}



The set of independent variables we will use in our computations is $(\rho,Y,T)$,

where $Y= Y_n-Y_p =\frac{\rho_n-\rho_p}{\rho}$:
\begin{equation}
 \begin{split}
  \partial_{\rho}X =& \partial_{\rho_n}X + \partial_{\rho_p}X, \\
  \partial_{Y}X=& \frac{1}{\rho} (\partial_{\rho_n}X - \partial_{\rho_p}X)
 \end{split}
\end{equation}

Thus,
\begin{equation}
 \begin{split}
  \partial_{\rho}M^{*}_r=&-\frac{2{M_r^{*}}^2}{\hbar^2}(\alpha_1+\alpha_2)=-2\frac{{M^{*}_r}^2}{M}F\\
  \partial_{\rho}U_r=&4 (A+B)+2 C \delta(1+\delta)\rho^{\delta-1}+4D\rho^{\delta-3}\big[\delta \rho_r^2+(3\delta-2)\rho_{-r}^2+2[1+\delta(\delta-1)\rho_r\rho_{-r}]\big]\\
  &+(\alpha_1+\alpha_2)\partial_{\rho}(\tau_r+\tau_{-r})\\
  =&4 (A+B)+2 C \delta(1+\delta)\rho^{\delta-1}+4D\rho^{\delta-3}\big[\delta \rho_r^2+(3\delta-2)\rho_{-r}^2+2[1+\delta(\delta-1)\rho_r\rho_{-r}]\big]\\
  &+\sum_r\big[3TG_r\frac{M_r^{*}}{\hbar^2}+F(\frac{9M_r^{*}}{\hbar^2}T\rho_rG_r-5\tau_r)\big]\\
  \partial_{\rho}\tau_r=&\frac{3G_rM_r^{*}T}{\hbar^2}(1+3F\rho_r)-5F \tau_r\\
  \partial_{\rho}\eta_r=&\frac{G_r}{\rho_r}+3F\rho_r\\
  \partial_{Y}M^{*}_r=&-\text{sgn}(r)\frac{2M_r^{*}}{\rho \hbar^2}(\alpha_1-\alpha_2),\ \text{sgn}(n)=-1,\ \text{sgn}(p)=1\\
  \partial_Y U_r=& \text{sgn}(r)\big[4 \frac{B}{\rho}-4D \rho^{\delta-3}[(\delta-2)\rho_p-\delta \rho_n]\big]+\alpha_1\frac{\partial_Y\tau_r}{\rho}+\alpha_2\frac{\partial_Y\tau_{-r}}{\rho}\\
  =&\text{sgn}(r)\big[4 \frac{B}{\rho}-4D \rho^{\delta-3}[(\delta-2)\rho_p-\delta \rho_n]-\frac{3 G_r M_r T}{\rho \hbar^2}+G(\frac{G_r M_r Y_r T}{ \hbar^2}-5\tau_r)\big],\ Y_r = \frac{1-\text{sgn}(r)Y}{2}\\
 \partial_{Y}\tau_r=&-\text{sgn}(r)\big[\frac{3G_rTM_r}{\hbar^2}(1-3G\rho_r)-5G\tau_r\big]\\
 \partial_Y\eta_r=&-\text{sgn}(r)\big[\frac{G_r}{\rho_r}-3G\rho_r\big]\\
 \end{split}
\end{equation}
From the expressions above the derivatives of the $\mu_r$ can be derived:
\begin{equation}
 \begin{split}
  \partial_{\rho}\mu_r=\frac{TG_r}{\rho_r}(1+ 3\rho_r F)+\partial_{\rho}U_r,\\
  \partial_{Y}\mu_r=-\text{sgn}(r)\frac{TG_r}{\rho \rho_r}(1-3\rho_r G)+\partial_{Y}U_r&\\
 \end{split}
\end{equation}
And, the chemical potential we will use is $\mu = \mu_n-\mu_p$:
\begin{equation}
 \begin{split}
  \partial_\rho \mu=&\partial_{\rho_n} \mu+\partial_{\rho_p} \mu\\
  =&\frac{5 FG\hbar^2}{M}(\tau_n-\tau_p)+\frac{3GT}{M}\Big[G_p M^{*}_p (1+3F\rho_p)-G_n M^{*}_n(1+3F\rho_n)\Big]\\
  &+3FT(G_n-G_p)+\frac{2T}{(1-Y^2)\rho}[(G_n-G_p)-(G_n+G_p)Y]-8D(\delta-1)Y\rho^{\delta-1}\\
  \partial_{\rho Y} \mu=&\partial_Y\Big[\frac{5 FG\hbar^2}{M}(\tau_n-\tau_p)+\frac{3GT}{M}\Big[G_p M^{*}_p (1+3F\rho_p)-G_n M^{*}_n(1+3F\rho_n)\Big]\\
  &+3FT(G_n-G_p)+\frac{2T}{(1-Y^2)\rho}[(G_n-G_p)-(G_n+G_p)Y]-8D(\delta-1)Y\rho^{\delta-1}\Big]\\
  &=\frac{1}{\rho}\sum_r\Big[\frac{{G_r} T}{{\rho_r}^2} \big[{J_r}+3 F (1+{J_r}) {\rho_r}\big]\\
  &+3G\frac{ {G_r} {M_r} T}{M \rho_r} (1+{J_r}+F {\rho_r}+3 F {J_r} {\rho_r})+3 GT(1+J_r+3 FJ_r\rho_r+3 F\rho_r)\\
  &-3 G^2 G_r \frac{M_r}{M}T\big[15 F \rho_r+2\frac{M_r^{*}}{M}(1+3F\rho_r)\big]+9G^2T(1+J_r)\frac{M_r^{*}}{M}\rho_r(1+3 F \rho_r)+G^2F\frac{25 \hbar^2\tau_r}{M}\Big]\\
  &-16 d (\delta -1) \rho^{\delta -3}\\
  \partial_T\mu =&\partial_{T} \mu_n-\partial_{T} \mu_p\\
  =&(\eta_n-\eta_p)-\frac{3}{2}(G_n-G_p)+\frac{9}{2}G(G_n \rho_n -G_p \rho_p)-\frac{5 G \hbar^2}{T} (\frac{\tau_n}{M^{*}_n}-\frac{\tau_p}{M^{*}_p})\\
  \end{split}
\end{equation}

For higher order thermodynamic derivatives, the following calculation will be needed:
\begin{equation}
 \begin{split}
   \partial_{\eta_r}G_r(\eta_r)=&1+J_r(\eta_r)\\
   J_r(\eta_r)=&\frac{F_{-3/2}(\eta_r)F_{1/2}(\eta_r)}{F_{-1/2}(\eta_r)}
 \end{split}
\end{equation}


The entropy per baryon is calculated in the previous section and it is $\rho s=\sum_r(\frac{5 \hbar^2\tau_r}{6M_r^{*}T}-\rho_r \eta_r)$:
\begin{equation}
 \begin{split}
 \rho \partial_{T}s=&\sum_r \frac{5 \hbar^2}{6M_r^{*}T}(\partial_T\tau_r-\tau_r/T)-\rho_r\partial_T\eta_r)\\
 =&\frac{1}{4T}\sum_r(\frac{5\hbar^2\tau_r}{TM_r^{*}}-9\rho_r G_r)\\ 
 s+\rho \partial_{\rho}s=&\sum_r\Big[\frac{5 \hbar^2}{6M_r^{*}T}(\partial_{\rho}\tau_r-\tau_r \partial_{\rho}M^{*}_r/M_r^{*})-y_r\eta_r-\rho_r\partial_{\rho}\eta_r)\Big]\\
 =&\sum_r\Big[\frac{3}{2}G_r+3F\rho_r(\frac{5}{2}G_r-\rho_r)+\frac{5\hbar^2\tau_r}{6M_r^{*}T}F(2\frac{M_r^{*}}{M}-5)-y_r\eta_r\Big]\leftrightarrow \\
 \rho \partial_{\rho}s=&\sum_r\Big[\frac{3}{2}G_r(1+5 F \rho_r)-3 F \rho_r^2-\frac{5\hbar^2\tau_r}{6M T}\big[\frac{M}{M_r^{*}}(\frac{1}{\rho}+5 F)-2F\big]\Big]\leftrightarrow \\
 \partial_{\rho}s+\rho \partial_{\rho \rho}s=&\sum_r\Big[\frac{3}{2} \big[{G_r}+5 F {G_r} {y_r}-4 F {y_r} {\rho_r}+(1+{J_r}) (\frac{{G_r}}{{\rho_r}}+3 F {\rho_r}) (1+5 F {\rho_r})\big]\\
 &+\frac{5 \hbar^2\tau_r}{6 M T}\big[5 F^2 (5 \frac{M_r^{*}}{M}-4)+\frac{F}{\rho}(5 \frac{M}{M_r^{*}}-2)+\frac{M}{M_r^{*}\rho^2}\big]-\frac{5}{2}\frac{M_r^{*}}{M}G_r(1+3 F \rho_r)\big[\frac{M}{M_r^{*}}(\frac{1}{\rho}+5 F)-2F\big]  \Big] \\
 \rho \partial_{\rho \rho}s =& \sum_r \Big[\frac{3}{2} {G_r} (1+5 F {y_r})-\frac{3}{2} {G_r}(5 F {y_r}+\frac{1}{\rho })-3 F {y_r} {\rho_r}+\frac{3}{2} (1+{J_r}) (\frac{{G_r}}{{\rho_r}}+3 F {\rho_r}) (1+5 F {\rho_r})\\
 &-\frac{5}{2} G_r (1+ 3 F \rho_r)(\frac{1}{\rho}+5 F-2 F \frac{M_r^{*}}{M})+\frac{5 \hbar^2 \tau_r}{6 M_r^{*} T \rho^2}\big[2-4 F \frac{M_r^{*}}{M}\rho (1+5 F \rho)+5 F \rho (2 + 5 F \rho)\big]\Big]\\
 \rho \partial_{T\rho}s=&\frac{3}{4T}\sum_r\Big[(2-4y_r-J_r)G_r+3\rho_r F\big[5G_r-3(1+J_r)\rho_r\big]+\frac{5\hbar^2\tau_r}{TM_r^{*}}\big[F(2\frac{M^{*}_r}{M}-5)-\frac{1}{3\rho}\big]\Big]\\
-\partial_{TT}s=&\partial_{T}\Big[\frac{1}{4T\rho}\sum_{r={n,p}}(\frac{5\hbar^2\tau_r}{T M_r^{*}}-9\rho_r G_r)\Big]\\
  =&\frac{1}{4T\rho}\sum_{r={n,p}}\big[\frac{5\hbar^2(\partial_T\tau_r-2\tau_r/T)}{TM_r^{*}}-9\rho_r (\partial_TG_r-G_r/T)\big]\\
  =&\frac{1}{4T\rho}\sum_{r={n,p}}\big[\frac{25\hbar^2\tau_r}{2M_r^{*}T^2}-9\rho_r (\frac{5}{2}G_r/T+\partial_TG_r-G_r/T)\big]\\
  =&\frac{1}{4T\rho}\sum_{r={n,p}}\big[\frac{15\hbar^2\tau_r}{2M_r^{*}T}-9\rho_r \big[\frac{5}{2}G_r/T-G_r/T-\frac{3}{2}G_r/T(1+J_r)]\big]\\
  =&\frac{3}{8T^2\rho}\sum_{r={n,p}}\big[\frac{5\hbar^2\tau_r}{M_r^{*}}-9\rho_rG_r J_r\big]\\
 \end{split}
\end{equation}

Thus, we can give analytical expressions for all independent $2^{nd}$ derivatives of the free energy per baryon (see NSE notes). The 
notation for number density is $n$ in NSE notes and $\rho$ so far here, everything else is the same. Here we switch to match NSE notation.

\begin{mdframed}
 \begin{equation}
  \begin{split}
   \partial_{TT}f=&-\partial_Ts=-\frac{1}{4Tn}\sum_{r={n,p}}(\frac{5\hbar^2\tau_r}{TM_r^{*}}-9n_r G_r)\\
   \partial_{nY}f=&\partial_n\mu=\frac{5 FG\hbar^2}{M}(\tau_n-\tau_p)+\frac{3GT}{M}\Big[G_p M^{*}_p (1+3Fn_p)-G_n M^{*}_n(1+3Fn_n)\Big]\\
  &+3FT(G_n-G_p)+\frac{2T}{(1-Y^2)n}[(G_n-G_p)-(G_n+G_p)Y]-8D(\delta-1)Yn^{\delta-1}\\
  \partial_{nT}f=& s/n+\partial_T\mu=\frac{1}{n^2}\sum_r(\frac{5\hbar^2 \tau_r}{6 M_r^{*}T}-n_r\eta_r)+(\eta_n-\eta_p)-\frac{3}{2}(G_n-G_p)\\
  &-\Big[\frac{5\hbar^2}{2T}[(\frac{\tau_n}{M_n^{*}}-\frac{\tau_p}{M_p^{*}})-\frac{9}{2}(n_n \tau_n-n_p \tau_p)]\Big]
  \end{split}
 \end{equation}
\end{mdframed}

For $3^{rd}$ order derivatives, the expressions above in connections with NSE notes can be used to obtain analytical results:
\begin{mdframed}
\begin{equation}
 \begin{split}
  \partial_{TTT}f=&-\partial_{TT}s\\
=&\frac{3}{8T^2\rho}\sum_{r={n,p}}\big[\frac{5\hbar^2\tau_r}{M_r^{*}}-9\rho_rG_r J_r\big]\\
  \partial_{\{TTn\}}f=&-\partial_{\{Tn\}}s\\
  =&-\frac{3}{4T n}\sum_r\Big[(2-4y_r-J_r)G_r+3n_r F\big[5G_r-3(1+J_r)n_r\big]+\frac{5\hbar^2\tau_r}{TM_r^{*}}\big[F(2\frac{M^{*}_r}{M}-5)-\frac{1}{3n}\big]\Big]\\
  \partial_{\{nnT\}}f=&-\partial_{nn}s\\
  =&-\frac{1}{n}\sum_r \Big[\frac{3}{2} {G_r} (1+5 F {y_r})-\frac{3}{2} {G_r}(5 F {y_r}+\frac{1}{n })-3 F {y_r} {n_r}+\frac{3}{2} (1+{J_r}) (\frac{{G_r}}{{n_r}}+3 F {n_r}) (1+5 F {n_r})\\
 &-\frac{5}{2} G_r (1+ 3 F n_r)(\frac{1}{n}+5 F-2 F \frac{M_r^{*}}{M})+\frac{5 \hbar^2 \tau_r}{6 M_r^{*} T n^2}\big[2-4 F \frac{M_r^{*}}{M}n (1+5 F n)+5 F n (2 + 5 F n)\big]\Big]\\
 \partial_{\{nnY\}}f=&\partial_{\{nnY\}}\mu=\frac{1}{\rho}\sum_r\Big[\frac{{G_r} T}{{\rho_r}^2} \big[{J_r}+3 F (1+{J_r}) {\rho_r}\big]\\
  &+3G\frac{ {G_r} {M_r} T}{M \rho_r} (1+{J_r}+F {\rho_r}+3 F {J_r} {\rho_r})+3 GT(1+J_r+3 FJ_r\rho_r+3 F\rho_r)\\
  &-3 G^2 G_r \frac{M_r}{M}T\big[15 F \rho_r+2\frac{M_r^{*}}{M}(1+3F\rho_r)\big]+9G^2T(1+J_r)\frac{M_r^{*}}{M}\rho_r(1+3 F \rho_r)+G^2F\frac{25 \hbar^2\tau_r}{M}\Big]\\
  &-16 d (\delta -1) \rho^{\delta -3}\\
 \end{split}
\end{equation} 
\end{mdframed}







\end{document}
